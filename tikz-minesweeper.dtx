% \iffalse meta-comment
% !TeX program  = XeLaTeX
% !TeX encoding = UTF-8
% Copyright (C) 2021-2023 by Tian-Yi Pu, Fei-Yu Xiang
% and Yao-Yu Zhu
%
% This file may be distributed and/or modified under the
% conditions of the LaTeX Project Public License, either
% version 1.3 of this license or (at your option) any later
% version. The latest version of this license is in:
%
%    http://www.latex-project.org/lppl.txt
%
% and version 1.3 or later is part of all distributions of
% LaTeX version 2005/12/01 or later.
%
% \fi

% \iffalse
%<package>\NeedsTeXFormat{LaTeX2e}[2005/12/01]
%<package>\ProvidesPackage{tikz-minesweeper}
%<package>    [2023/04/23 v0.2.2 Draw a minesweeper board in LaTeX]
%<*batchfile>
\begingroup
\input ctxdocstrip.tex
\keepsilent
\usedir{tex/latex/tikz-minesweeper}

\preamble
Copyright (C) 2021-2023 by Tian-Yi Pu, Fei-Yu Xiang
and Yao-Yu Zhu

This file may be distributed and/or modified under the
conditions of the LaTeX Project Public License, either
version 1.3 of this license or (at your option) any later
version. The latest version of this license is in:

   http://www.latex-project.org/lppl.txt

and version 1.3 or later is part of all distributions of
LaTeX version 2005/12/01 or later.
\endpreamble

\postamble
This package consists of the files tikz-minesweeper.dtx,
                                   README.md,
             and the derived files tikz-minesweeper.sty,
                                   tikz-minesweeper.pdf.
\endpostamble

\generate{\file{tikz-minesweeper.sty}{\from{\jobname.dtx}{package}}}

\obeyspaces
\typeout{****************************************************}
\typeout{*                                                  *}
\typeout{* To finish the installation you have to move the  *}
\typeout{* following file into a directory searched by TeX: *}
\typeout{*                                                  *}
\typeout{*     tikz-minesweeper.sty                         *}
\typeout{*                                                  *}
\typeout{* To produce the documentation run the file        *}
\typeout{* tikz-minesweeper.dtx through LaTeX.              *}
\typeout{*                                                  *}
\typeout{* Happy TeXing (and minesweeping) !                *}
\typeout{*                                                  *}
\typeout{****************************************************}

\endgroup
%</batchfile>
%<*driver>
\documentclass{ctxdoc}
\usepackage{tikz-minesweeper}
\usepackage{verbatim}
\usepackage{booktabs}
\usepackage{colortbl}
\usepackage{subfig}
\usepackage{float}
\usepackage{multicol}

\ctexset{
    section = {
        name = {,},
        number = \arabic{section},
    }
}
\EnableCrossrefs
\CodelineIndex
\begin{document}
    %%
%% This is file `zh.tex',
%% Chinese version of the `tikz-minesweeper' package
%%
%% Copyright (C) 2021-2023 by Tian-Yi Pu, Fei-Yu Xiang
%% and Yao-Yu Zhu
%%
%% This file may be distributed and/or modified under the
%% conditions of the LaTeX Project Public License, either
%% version 1.3 of this license or (at your option) any later
%% version. The latest version of this license is in:
%%
%%    http://www.latex-project.org/lppl.txt
%%
%% and version 1.3 or later is part of all distributions of
%% LaTeX version 2005/12/01 or later.

\renewcommand{\figurename}{图}
\renewcommand{\tablename}{表}
\renewcommand{\indexname}{索引}
\def\TransIndexPrologueContent{\emph{意大利体的数字表示描述对应索引项的页码, 带下划线的数字表示定义对应索引项的代码行号, 罗马字体的数字表示使用对应索引项的代码行号.}}

\def\TransTitle{\pkg{tikz-minesweeper}宏包使用手册 (\fileversion)}
\def\TransAuthor{濮天羿~~~~向飞宇\thanks{源代码请参考此仓库: https://github.com/T0nyX1ang/tikz-minesweeper}~~~~朱耀宇}

\def\TransSectionUsage{绘制示例}
\def\TransSectionUsageContent{下面是一个\pkg{tikz-minesweeper}宏包的绘制示例, 包含了该宏包中最重要的几个指令.}
\def\TransFigureUsageCaption{在5x5区域内绘制数字、雷、旗帜、空格和标记}

\def\TransSectionCommand{绘制指令介绍}

\def\TransSubsectionFundamentals{基础设定}
\def\TransSubsectionFundamentalsUnitContent{本宏包中设定单位长度为\textbf{1pt}. 为了与标准扫雷游戏一致, 一格的边长是\textbf{16pt}. 下文如无特别说明, 所有方格的原点都位于左上角.}
\def\TransSubsectionFundamentalsScaleContent{如果需要缩放一个盘面, 推荐使用scope环境中的scale参数, 即采用如下语法:}
\def\TransFigureZoomedCellCaption{两倍放大一个格子}

\def\TransSubsectionColorScheme{配色方案}
\def\TransSubsectionColorSchemeContent{\pkg{tikz-minesweeper}宏包定义了用于绘制扫雷数字和盘面的配色. 这些配色均给出了标签, 这些标签均可以直接在文档中使用.}
\def\TransSubsectionColorSchemeTableContent{
    \begin{tabular}{ccccc}
        \toprule
        颜色标签           & R   & G   & B   & 场景                    \\
        \midrule
        \ColorTagZero{}   & 192 & 192 & 192 & 格子内部 / 外边框中间层   \\
        \ColorTagOne{}    & 0   & 0   & 255 & 数字1                   \\
        \ColorTagTwo{}    & 0   & 128 & 0   & 数字2                   \\
        \ColorTagThree{}  & 255 & 0   & 0   & 数字3                   \\
        \ColorTagFour{}   & 0   & 0   & 128 & 数字4                   \\
        \ColorTagFive{}   & 128 & 0   & 0   & 数字5                   \\
        \ColorTagSix{}    & 0   & 128 & 128 & 数字6                   \\
        \ColorTagSeven{}  & 0   & 0   & 0   & 数字7                   \\
        \ColorTagEight{}  & 128 & 128 & 128 & 数字8                   \\
        \ColorTagBorder{} & 160 & 160 & 160 & 边框阴影                 \\
        \ColorTagLight{}  & 255 & 255 & 255 & 与阴影对比的高亮颜色      \\
        \ColorTagShade{}  & 128 & 128 & 128 & 格子阴影 / 按下的格子边界  \\
        \ColorTagFail{}   & 255 & 0   & 0   & 导致失败的格子            \\
        \bottomrule
    \end{tabular}
}
\def\TransSubsectionColorSchemeTableCaption{颜色对照表}

\def\TransSubsectionCellElements{方格元素}
\def\TransCommandCellContent{在第 \meta{r} 行第 \meta{c} 列交叉处绘制包含内容 \meta{info} 的方格, 行列均从 0 开始编号. \meta{info} 的检查顺序如下:
    \begin{optdesc}
        \item[0] 如果 \meta{info} 是 \meta{0}, 仅绘制一个按下状态的方格. 按下状态的方格首先用 \ColorTagShade{} 绘制方格边界, 然后用 \ColorTagZero{} 绘制底色, 左和上的边界宽度均为 1pt, 此时方格的中心点为 (8, -8);
        \item[1{\textbar}2{\textbar}3{\textbar}4{\textbar}5{\textbar}6{\textbar}7{\textbar}8] 如果 \meta{info} 是 \meta{1}-\meta{8}, 方格是按下状态, 并且填入带配色的矢量化数字;
        \item[r{\textbar}g{\textbar}b{\textbar}c{\textbar}y{\textbar}o{\textbar}v] 如果 \meta{info} 是 \meta{r} (red, 红色)、\meta{g} (绿色)、\meta{b} (蓝色)、\meta{c} (cyan, 湖蓝色)、\meta{y} (yellow, 黄色)、\meta{o} (orange, 橘色)、\meta{v} (violet, 紫色)中的一个, 方格是抬起状态, 且会染成对应颜色, 方格颜色的透明度为 0.2;
        \item[f] 如果 \meta{info} 是 \meta{f}, 方格是弹起状态并且会插旗. 旗帜首先用 \ColorTagSeven{} 绘制两层旗子的底座, 然后用 \ColorTagSeven{} 绘制旗子的旗杆, 最后用 \ColorTagThree{} 绘制旗帜本身;
        \item[m] 如果 \meta{info} 是 \meta{m}, 方格是按下状态并且会显示地雷. 地雷首先用 \ColorTagSeven{} 绘制地雷的梯形突起, 然后用 \ColorTagSeven{} 绘制地雷的圆形本体与本体上的光影;
        \item[s] 如果 \meta{info} 是 \meta{s}, 方格是按下状态并且会显示半透明的地雷, 即将地雷的颜色调整为 \ColorTagShade{};
        \item[n] 如果 \meta{info} 是 \meta{n}, 方格是按下状态并且会显示踩到地雷, 用 \ColorTagFail{} 绘制底色;
        \item[e] 如果 \meta{info} 是 \meta{e}, 方格是按下状态并且会显示标记错误的地雷, 即在普通地雷的基础上, 进一步用 \ColorTagFail{} 绘制一个叉;
        \item[-] 如果 \meta{info} 是 \meta{-}, 方格是弹起状态并且无字符. 弹起状态的方格首先分别用 \ColorTagLight{} 和 \ColorTagShade{} 对称地绘制亮面和暗面, 营造立体感, 亮面和暗面的宽度均为 2pt, 然后用 \ColorTagZero{} 绘制底色. 需要注意的是, 该方格的中心点为 (8.5, -8.5);
        \item[(?)] 如果 \meta{info} 是其他\textbf{单个}字符, 方格是弹起状态并包含该字符, 字符中心为 (8, -8);
        \item[(*)] 如果 \meta{info} 包含\textbf{多个}字符, 将会自动缩小字符, 转化为 \tn[no-index]{tiny} 模式.
    \end{optdesc}
    相关元素绘制细节见图\ref{fig-flag-mine}和图\ref{fig-cellup-celldown}.
}
\def\TransFigureFlagMineNumCaption{旗子、雷和数字的细节图}
\def\TransFigureCellUpDownCaption{弹起的方格与按下的方格 (包含中心点)}

\def\TransSubsectionFrameElements{边框元素}
\def\TransCommandBorderContent{绘制以格为单位长度(宽度)的边框. \meta{t}、\meta{l}、\meta{b}、\meta{r} 分别代表上、左、下、右边框, 如果相邻边框同时绘制, 将自动绘制对应的角. 边框分为三层, 中间层使用 \ColorTagZero{} 绘制, 宽度为 6pt, 内外层用 \ColorTagLight{} 与 \ColorTagBorder{} 提供立体感, 宽度为 3pt. 提供立体感的方式和弹起方格同理. 左边和上边的高亮外边缘用 \ColorTagBorder{} 封边. 默认参数为 \meta{-}, 即绘制全部边框.}
\def\TransFigureBorderCaption{边框组装效果}

\def\TransSubsectionModules{综合模块}
\def\TransSubsectionModulesContent{这一部分的模块是 \pkg{tikz-minesweeper} 提供给用户使用的模块, 如果没有特定的设计需求, 推荐仅使用这一部分提供的模块.}
\def\TransCommandRowContent{在第 \meta{r} 行从左到右绘制多个方格. \meta{seq:info} 为一个含有 \meta{info} 的序列, 序列中的每个元素作为 \tn{cell} 的第三个参数. 如果希望输入多个字符, 需要使用大括号, 例如 \meta{1\{23\}4} 的序列长度为3, 该序列中的元素分别为 \meta{1}, \meta{23} 和 \meta{4}.}
\def\TransCommandColContent{在第 \meta{c} 列从上到下绘制多个方格, 语法与 \tn{row} 一致.}
\def\TransCommandBoardContent{绘制 \meta{r} 行 \meta{c} 列的边框, \meta{t}, \meta{r}, \meta{b}, \meta{l} 为与 \tn{border} 对应可叠加的边框开关, 默认全部开启, 即地图中有完整边框, 如果使用了 \meta{-} 标志, 代表去除某些特定方位的边框. 此选项可以根据自己需求进行定制. 解析器只会解析最后一个 \meta{-} 标志的位置, 即 \tn[no-index]{border[-br-tl-lb-b]} 会被解析为 \tn[no-index]{border[-b]}. 解析器不会解析重复的参数, 即 \tn[no-index]{border[ttblbtlb]} 会被解析为 \tn[no-index]{border[tlb]}. 可选指令 \meta{x} 表示是否显示横纵坐标, 默认为不显示横纵坐标.}
\def\TransCommandBoardExtraNoteContent{在绘制边框的过程中, 该指令会根据边框类型, 利用 \pkg{tikz} 宏包的截取指令, 自动以 \ColorTagBorder{} 为颜色, 绘制一个 0.2pt 的外边框. 同时, 为了避免格子的同色邻接区域导致的渲染问题, 该指令会提前绘制同色邻接区域.}
\def\TransFigureBoardOptionCaption{边框指令示例}
\def\TransFigureBoardCoordinateCaption{横纵坐标显示示例}
\def\TransCommandColorCellContent{将 \meta{seq: pos} 位置的方格染色为 \meta{color} 颜色. \meta{color} 仅能传入 \tn{cell} 中接受的颜色类型, 但是与 \tn{cell} 仅支持单一颜色不同的是, \meta{color} 支持填入至多 4 种不同颜色, 颜色将从方格顶部逆时针填充. \meta{seq: pos} 定义了一个位置序列, 格式如下:
    \begin{itemize}
        \item 序列中的每个元素以分号(;)分隔.
        \item 对于每个元素, 以逗号(,)分隔行区间和列区间.
        \item 对于每个区间, 如果它是\textbf{单字符}, 则表示区间的起始位置和终止位置相同, 均为该字符; 否则需要以减号(-)分隔起始位置和终止位置.
    \end{itemize}
}
\def\TransFigureColorCellCaption{方格染色示例}

\def\TransSectionImplementation{代码实现}
\def\TransMacroDependencyContent{定义 \pkg{tikz-minesweeper} 宏包的依赖, 分别为 \LaTeXiii{} 和 \pkg{tikz} 宏包.}
\def\TransMacroColorSchemeContent{定义 \pkg{tikz-minesweeper} 宏包的配色方案. 在实际使用该宏包时, 可以使用类似方法定义自己的配色方案.}
\def\TransMacroUnitContent{定义 \pkg{tikz-minesweeper} 宏包的基本单位, 即 1pt.}
\def\TransMacroEnableLaTeXiiiEnvContent{打开\LaTeXiii{}编程环境.}
\def\TransMacroVariableDefinitionContent{定义 \pkg{tikz-minesweeper} 宏包中的变量.}
\def\TransMacroValidateInRangeContent{判断 \meta{\#1} 是否介于下界 \meta{\#2} 和上界 \meta{\#3} 之间, 包含边界. 如果下界 \meta{\#2} 为空值, 则仅判断 \meta{\#1} 是否小于上界 \meta{\#3}; 如果上界 \meta{\#3} 为空值, 则仅判断 \meta{\#1} 是否大于上界 \meta{\#2}.}
\def\TransMacroFlagContent{绘制旗子.}
\def\TransMacroMineContent{绘制地雷.}
\def\TransMacroMineShadedContent{绘制半透明的地雷. 需要注意的是, 半透明的雷的颜色没有与其它颜色解耦, 如果定义了不同的配色方案, 请谨慎绘制这种地雷.}
\def\TransMacroCellUpContent{绘制鼠标弹起时的方格.}
\def\TransMacroCellDownContent{绘制鼠标按下后的方格.}
\def\TransMacroCellFailContent{绘制由于踩雷而失败的方格.}
\def\TransMacroMisflagFailContent{绘制由于标记错误而失败的方格.}
\def\TransMacroCellNumContent{用 \ColorTagOne{} 至 \ColorTagEight{} 绘制给定的矢量化数字 1 至 8, 接受一个参数 \meta{\#1}. 参数 \meta{\#1} 仅接受 \meta{12345678} 中的字符, 传入其它值将会被忽略.}
\def\TransMacroCellColoredContent{绘制鼠标按下时的染色方格, 接受一个参数 \meta{\#1}. 参数 \meta{\#1} 仅接受 \meta{rgbcyov} 中的字符, 传入其它值将会被忽略. 参数 \meta{\#1} 的最长长度为 4, 如果传入的参数长度小于 4, 则会在参数的左/右侧补充与其相同的字符, 如果传入的参数长度大于 4, 则会直接被截断为 4 个字符.}
\def\TransMacroCellContent{绘制方格, 接受三个参数 \meta{\#1}, \meta{\#2}, \meta{\#3}. 在坐标(\meta{\#3}, \meta{\#2})处绘制一个方格, 方格中的内容由 \meta{\#1} 决定, \meta{\#1}可以传入任意参数. 该函数依次检查 \meta{\#1} 是否为数字参数 \meta{012345678}, 颜色参数 \meta{rgbcyov}, 雷与旗帜参数 \meta{fmsne-}, 除上述字符外的单字符参数以及多字符参数.}
\def\TransMacroLineContent{绘制一行或者一列的格子, 接受三个参数 \meta{\#1}, \meta{\#2}, \meta{\#3}. \meta{\#1} 为行号或者列号, \meta{\#2} 为格子内容, \meta{\#3} 为按行 \meta{r} 或者按列 \meta{c} 依次绘制格子, 其余参数将不会绘制.}
\def\TransMacroRowColMarkerContent{绘制行列坐标数字.}
\def\TransMacroGapFillerContent{由于渲染问题, 绘制盘面时, 同色区域之间可能会出现一个空白的缝隙, 该函数用于提前填充这些缝隙.}
\def\TransMacroBorderContent{绘制边框, 接受三个参数 \meta{\#1}, \meta{\#2}, \meta{\#3}. \meta{\#1} 为边框种类参数, \meta{\#2} 为行数, \meta{\#3} 为列数. 参数 \meta{\#1} 仅接受 \meta{-tlbr} 中的值.}
\def\TransMacroBoardContent{绘制盘面, 接受三个参数 \meta{\#1}, \meta{\#2}, \meta{\#3}. \meta{\#1} 为边框种类参数, \meta{\#2} 为行数, \meta{\#3} 为列数. 参数 \meta{\#1} 仅接受 \meta{-tlbrx} 中的值.}
\def\TransMacroColorCellContent{给多个区域染色, 接受两个参数 \meta{\#1}, \meta{\#2}. \meta{\#1} 仅接受 \cs[no-index]{msweeper_cellcolored:n} 中的颜色, \meta{\#2} 传入待绘制的多个区域.}
\def\TransMacroUserAPIContent{提供不显式调用 \LaTeXiii{} 的用户接口.}
\def\TransMacroDisableLaTeXiiiEnvContent{关闭 \LaTeXiii{} 编程环境.}

    %%
%% This is file `zh.tex',
%% English version of the `tikz-minesweeper' package's documents
%%
%% Copyright (C) 2021-2023 by Tian-Yi Pu, Fei-Yu Xiang
%% and Yao-Yu Zhu
%%
%% This file may be distributed and/or modified under the
%% conditions of the LaTeX Project Public License, either
%% version 1.3 of this license or (at your option) any later
%% version. The latest version of this license is in:
%%
%%    http://www.latex-project.org/lppl.txt
%%
%% and version 1.3 or later is part of all distributions of
%% LaTeX version 2005/12/01 or later.

\renewcommand{\figurename}{Figure}
\renewcommand{\tablename}{Table}
\renewcommand{\indexname}{Index}
\def\TransIndexPrologueContent{\emph{The italic numbers denote the pages where the corresponding entry is described, numbers underlined point to the definition, all others indicate the places where it is used.}}

\def\TransTitle{\pkg{tikz-minesweeper} Manual (\fileversion)}
\def\TransAuthor{Tian-Yi Pu~~~~Fei-Yu Xiang \thanks{Source code: https://github.com/T0nyX1ang/tikz-minesweeper}~~~~Yao-Yu Zhu}

\def\TransSectionUsage{Usage}
\def\TransSectionUsageContent{An example including the most important commands within the \pkg{tikz-minesweeper} package is as follows.}
\def\TransFigureUsageCaption{Rendering numbers, mines, flags, cells and marks on a 5x5 board}

\def\TransSectionCommand{Reference of commands}

\def\TransSubsectionFundamentals{Fundamental rules}
\def\TransSubsectionFundamentalsUnitContent{This package sets the unit length to \textbf{1pt}. The edge length for cells is \textbf{16pt} to match the standard minesweeper game. Henceforth, without explicit note, the origin of a cell is its top-left corner.}
\def\TransSubsectionFundamentalsScaleContent{To scale a board, it's recommended to use the scale parameter from scope environment, that is: }
\def\TransFigureZoomedCellCaption{A zoomed cell}

\def\TransSubsectionColorScheme{Color scheme}
\def\TransSubsectionColorSchemeContent{The \pkg{tikz-minesweeper} package defines the colors for numbers and board margins in minesweeper. The colors have labels that can be directly executed in the documentation.}
\def\TransSubsectionColorSchemeTableContent{
    \begin{tabular}{ccccc}
        \toprule
        Color label       & R   & G   & B   & Use case                            \\
        \midrule
        \ColorTagZero{}   & 192 & 192 & 192 & cell interior/middle layer of frame \\
        \ColorTagOne{}    & 0   & 0   & 255 & number 1                            \\
        \ColorTagTwo{}    & 0   & 128 & 0   & number 2                            \\
        \ColorTagThree{}  & 255 & 0   & 0   & number 3                            \\
        \ColorTagFour{}   & 0   & 0   & 128 & number 4                            \\
        \ColorTagFive{}   & 128 & 0   & 0   & number 5                            \\
        \ColorTagSix{}    & 0   & 128 & 128 & number 6                            \\
        \ColorTagSeven{}  & 0   & 0   & 0   & number 7                            \\
        \ColorTagEight{}  & 128 & 128 & 128 & number 8                            \\
        \ColorTagBorder{} & 160 & 160 & 160 & frame shade                         \\
        \ColorTagLight{}  & 255 & 255 & 255 & highlight in contrast with shades   \\
        \ColorTagShade{}  & 128 & 128 & 128 & cell shade/border of opened cell    \\
        \ColorTagFail{}   & 255 & 0   & 0   & blasted cell                        \\
        \bottomrule
    \end{tabular}
}
\def\TransSubsectionColorSchemeTableCaption{Look-up table of colors}

\def\TransSubsectionCellElements{Cell Elements}
\def\TransCommandFlagContent{Draw the flag. Draw a 2-layer base of the flag with \ColorTagSeven{} first, then draw a flag pole with \ColorTagSeven{}. In the end, draw a flag with \ColorTagThree{}.}
\def\TransCommandMineContent{Draw a mine. Draw trapezoidal spikes with \ColorTagSeven{} first, then draw a circular body and highlight with \ColorTagSeven{}.}
\def\TransFigureFlagMineCaption{Sketch of the flag and the mine}
\def\TransCommandCellUpContent{Draw a covered cell. Draw the light side and dark side first, the widths of which are both 2pt, with \ColorTagLight{} and \ColorTagShade{} respectively to create a 3D effect. Then fill the background with \ColorTagZero{}. Note that the center of this cell is (8.5, -8.5), which is different from that of \tn{celldown}.}
\def\TransCommandCellDownContent{Draw a revealed cell. Draw the edges with \ColorTagShade{}, then fill the background with \ColorTagZero{}. The widths of the left and top edges are both 1pt. Note that the center of this cell is (8, -8), which is different from that of \tn{cellup}.}
\def\TransFigureCellUpDownCaption{Covered cell and revealed cell (the centers are marked)}
\def\TransCommandCellNumContent{Fill the cell with colored vectorized number~\raisebox{-0.25ex}{\tikz{\cellnum{1}} \tikz{\cellnum{2}} \tikz{\cellnum{3}} \tikz{\cellnum{4}} \tikz{\cellnum{5}} \tikz{\cellnum{6}} \tikz{\cellnum{7}} \tikz{\cellnum{8}}}. If \meta{num} is not from 1 to 8, then the command will not take effect. Note that this command is usually used with \tn{celldown}. It's not recommended using this command alone due to the general rule of minesweeper graphics. }

\def\TransSubsectionFrameElements{Frame elements}
\def\TransCommandBorderContent{Draw a cell-sized frame. \meta{t}, \meta{l}, \meta{b}, \meta{r} represent the top, left, bottom, right frame respectively. If two adjacent sides are drawn at the same time, the corner in between will also be drawn. The frame consists of 3 layers. The middle layer is in \ColorTagZero{} and 6pt wide. The inner and outer layer are 3pt wide, using \ColorTagLight{} and \ColorTagBorder{} to create a 3D effect similarly as \tn{cellup}. The outer edge of the left and top highlights are enclosed with \ColorTagBorder{}. The default argument is \meta{-}, which means drawing the full frame.}
\def\TransFigureBorderCaption{The full frame in effect}

\def\TransSubsectionModules{Modules}
\def\TransSubsectionModulesContent{This section contains the modules provided by \pkg{tikz-minesweeper}. In general, it's recommended to only use these commands unless there is specific customization needed.}
\def\TransCommandCellContent{Draw a cell containing \meta{info} at the intersection of the \meta{r}-th row and the \meta{c}-th column, starting from 0. \meta{info} is checked in the following order:
    \begin{itemize}
        \item If \meta{info} is a number from \meta{0} to \meta{8}, then the cell is pressed down and the number is colored;
        \item If \meta{info} is \meta{r} (red), \meta{g} (green), \meta{b} (blue), \meta{c} (cyan), \meta{y} (yellow), \meta{o} (orange) or \meta{v} (violet), then the cell is covered and accordingly colored with a transparency of 0.2;
        \item If \meta{info} is \meta{f}, then the cell is covered and flagged;
        \item If \meta{info} is \meta{m}, then the cell is pressed and shows a mine;
        \item If \meta{info} is \meta{s}, then the cell is pressed and shows a semitransparent mine, which is basically a mine in \ColorTagShade{};
        \item If \meta{info} is \meta{n}, then the cell is pressed and shows a blasted mine, with \ColorTagFail{} for background;
        \item If \meta{info} is \meta{e}, then the cell is pressed and shows a mistakenly flagged mine, which is a cross in \ColorTagFail{} on top of a mine;
        \item If \meta{info} is \meta{-}, then the cell is covered and has no content;
        \item If \meta{info} is any other \textbf{single} character, then the cell is covered and shows the character centered at (8, -8);
        \item If \meta{info} contains \textbf{multiple} characters, the string will be shown in \tn[no-index]{tiny} font.
    \end{itemize}
}
\def\TransCommandRowContent{Draw multiple cells from left to right in the \meta{r}-th row. \meta{seq:info} is a sequence containing \meta{info}, where each element in the sequence serves as the third parameter of \tn{cell}. To display multiple characters in a cell, these characters should be surrounded by curly braces. For example, the length of the sequence \meta{1\{23\}4} is 3, and the elements in this sequence are \meta{1}, \meta{23}, and \meta{4} respectively.}
\def\TransCommandColContent{Draw multiple cells from top to bottom in the \meta{c}-th column, and the syntax is the same as \tn{row}.}
\def\TransCommandBoardContent{Draw a border with \meta{r} rows and \meta{c} columns. \meta{t}, \meta{r}, \meta{b}, \meta{l} share the same syntax as \tn{border}. By default, the board will be surrounded by all borders. If the \meta{-} argument is used, the borders in specific directions will be removed, and this argument can be customized. The parser only parses the position of the last \meta{-} option. For example, \tn[no-index]{border[-br-tl-lb-b]} will be parsed as \tn[no-index]{border[-b]}. The parser does not parse repeated parameters. For example, \tn[no-index]{border[ttblbtlb]} will be parsed as \tn[no-index]{border[tlb]}. The argument \meta{x} indicates whether to display the coordinates, and the coordinates are hidden by default.}
\def\TransCommandBoardExtraNoteContent{During the process of drawing borders, this command will automatically use the clipping utility provided by \pkg{tikz}, based on the border type. Moreover, this command will draw an outer border with \ColorTagBorder{} and a thickness of 0.2pt. Additionally, to avoid rendering issues caused by adjacent areas of the same color, this command will pre-draw the adjacent areas of the same color.}
\def\TransFigureBoardOptionCaption{Examples for different border arguments}
\def\TransFigureBoardCoordinateCaption{Examples for displaying coordinates}
\def\TransCommandColorCellContent{Color the cells in a sequence of areas \meta{seq: pos} with the \meta{color}. The \meta{color} argument can only accept color types that are accepted by \tn{cell}. However, unlike \tn{cell}, which only supports a single color, the \meta{color} argument supports up to four different colors. The colors will be filled counterclockwise starting from the top of a cell. \meta{seq: pos} defines a sequence of positions with the following format:
    \begin{itemize}
        \item Each element in the sequence is separated by a semicolon (;);
        \item For each element, the row range and column range are separated by a comma (,);
        \item For each range, if it contains \textbf{only one} character, the starting and ending positions are the same; otherwise, these positions need to be separated by a hyphen (-).
    \end{itemize}
}
\def\TransFigureColorCellCaption{Examples for colored cells}

\def\TransSectionImplementation{Implementation}
\def\TransMacroDependencyContent{Define the dependencies of \pkg{tikz-minesweeper}, which are \LaTeXiii{} and \pkg{tikz}.}
\def\TransMacroColorSchemeContent{Define the color scheme of \pkg{tikz-minesweeper}. Customized color schemes can be defined similarly in practice.}
\def\TransMacroUnitContent{Define the basic unit of \pkg{tikz-minesweeper}, which is 1pt.}
\def\TransMacroEnableLaTeXiiiEnvContent{Enable the \LaTeXiii{} environment.}
\def\TransMacroVariableDefinitionContent{Define the variables in \pkg{tikz-minesweeper}.}
\def\TransMacroFlagContent{Draw a flag.}
\def\TransMacroMineContent{Draw a mine.}
\def\TransMacroMineShadedContent{Draw a shaded mine. Note that the color of a shaded mine is coupled with other colors in the package. Please draw shaded mines with caution if different color schemes are applied.}
\def\TransMacroCellUpContent{Draw a covered cell.}
\def\TransMacroCellDownContent{Draw a revealed cell.}
\def\TransMacroCellFailContent{Draw a failed cell by clicking a mine.}
\def\TransMacroMisflagFailContent{Draw a failed cell by misflaging.}
\def\TransMacroCellNumContent{Use \ColorTagOne{} to \ColorTagEight{} to draw the given vectorized numbers from 1 to 8. It accepts one parameter \meta{\#1}. The parameter \meta{\#1} only accepts characters from \meta{12345678}. Any other value passed will be ignored.}
\def\TransMacroCellColoredContent{Draw a colored revealed cell. It accepts one parameter \meta{\#1}, which only accepts characters from \meta{rgbcyov}. Any other value passed will be ignored. The maximum length of parameter \meta{\#1} is 4. If the length of the parameter is less than 4, it will be padded on the left/right side with the same character. If the length of the parameter is greater than 4, it will be truncated to 4 characters.}
\def\TransMacroCellContent{Draw a cell, accepting three parameters \meta{\#1}, \meta{\#2}, \meta{\#3}. The cell will be drawn at coordinates (\meta{\#3}, \meta{\#2}), and the content of the cell is determined by \meta{\#1}. \meta{\#1} can accept any parameter. The function sequentially checks if \meta{\#1} is a numeric parameter \meta{012345678}, a color parameter \meta{rgbcyov}, a mine or flag parameter \meta{fmsne-}, a single character parameter excluding the mentioned characters, or a multi-character parameter.}
\def\TransMacroLineContent{Draw a row or column of cells, accepting three parameters \meta{\#1}, \meta{\#2}, \meta{\#3}. \meta{\#1} represents the row number or column number, \meta{\#2} represents the content of the grids, and \meta{\#3} determines whether to draw the grids sequentially in a row (\meta{r}) or in a column (\meta{c}). Other parameters will not be drawn.}
\def\TransMacroRowColMarkerContent{Draw row and column coordinates.}
\def\TransMacroRawBorderContent{Draw the original border, and the parameters to draw a board will influence the final shape of the border.}
\def\TransMacroGapFillerContent{Due to rendering issues, when drawing a board, there may be gaps between adjacent areas of the same color. This function can pre-fill these gaps.}
\def\TransMacroBorderContent{Draw a border, accepting three parameters \meta{\#1}, \meta{\#2}, \meta{\#3}. \meta{\#1} represents the border type parameter, \meta{\#2} represents the number of rows, and \meta{\#3} represents the number of columns. The parameter \meta{\#1} accepts values from \meta{-tlbrx} only.}
\def\TransMacroColorCellContent{Color the cells in multiple areas, which accepts two parameters \meta{\#1} and \meta{\#2}. The parameter \meta{\#1} can only accept colors that are accepted by \cs[no-index]{msweeper_cellcolored:n}. The parameter \meta{\#2} is used to pass multiple regions to be drawn.}
\def\TransMacroUserAPIContent{Provide user APIs without calling \LaTeXiii{} directly.}
\def\TransMacroDisableLaTeXiiiEnvContent{Disable the \LaTeXiii{} environment.}

    \DocInput{tikz-minesweeper.dtx}
    \IndexLayout
    \IndexPrologue{%
    \section{\indexname}
    \textit{意大利体的数字表示描述对应索引项的页码,
    带下划线的数字表示定义对应索引项的代码行号,
    罗马字体的数字表示使用对应索引项的代码行号.}}
    \PrintIndex
\end{document}
%</driver>
% \fi
%
% \CheckSum{377}
%
% \CharacterTable
%  {Upper-case    \A\B\C\D\E\F\G\H\I\J\K\L\M\N\O\P\Q\R\S\T\U\V\W\X\Y\Z
%   Lower-case    \a\b\c\d\e\f\g\h\i\j\k\l\m\n\o\p\q\r\s\t\u\v\w\x\y\z
%   Digits        \0\1\2\3\4\5\6\7\8\9
%   Exclamation   \!     Double quote  \"     Hash (number) \#
%   Dollar        \$     Percent       \%     Ampersand     \&
%   Acute accent  \'     Left paren    \(     Right paren   \)
%   Asterisk      \*     Plus          \+     Comma         \,
%   Minus         \-     Point         \.     Solidus       \/
%   Colon         \:     Semicolon     \;     Less than     \<
%   Equals        \=     Greater than  \>     Question mark \?
%   Commercial at \@     Left bracket  \[     Backslash     \\
%   Right bracket \]     Circumflex    \^     Underscore    \_
%   Grave accent  \`     Left brace    \{     Vertical bar  \|
%   Right brace   \}     Tilde         \~}
%
% \GetFileInfo{tikz-minesweeper.sty}
%
% \title{\TransTitle}
% \author{\TransAuthor}
% \date{\filedate}
% \maketitle
%
% \section{\TransSectionUsage}
% \TransSectionUsageContent
%
% \begin{multicols}{2}
% \iffalse
%<*internal>
% \fi
\begin{verbatim}
    \usepackage{tikz-minesweeper}
    \begin{document}
        \begin{tikzpicture}
            \board[-x]{5}{5}
            \row{0}{A{20}12}
            \row{1}{se40}
            \row{2}{mn67}
            \row{3}{rgbc}
            \row{4}{ov-f}
            \col{4}{358y?}
        \end{tikzpicture}
    \end{document}
\end{verbatim}
% \iffalse
%</internal>
% \fi
%     \columnbreak
%     \begin{figure}[H]
%         \centering
%         \begin{tikzpicture}
%             \board[-x]{5}{5}
%             \row{0}{A{20}12}
%             \row{1}{se40}
%             \row{2}{mn67}
%             \row{3}{rgbc}
%             \row{4}{ov-f}
%             \col{4}{358y?}
%         \end{tikzpicture}
%         \caption{\TransFigureUsageCaption}
%     \end{figure}
% \end{multicols}
%
% \section{\TransSectionCommand}
%
% \subsection{\TransSubsectionFundamentals}
%
% \TransSubsectionFundamentalsUnitContent
%
% \TransSubsectionFundamentalsScaleContent
%
% \begin{multicols}{2}
% \iffalse
%<*internal>
% \fi
\begin{verbatim}
    \begin{tikzpicture}[scale=2]
        \board{1}{1}
        \row{0}{1}
    \end{tikzpicture}
\end{verbatim}
% \iffalse
%</internal>
% \fi
%     \columnbreak
%     \begin{figure}[H]
%         \centering
%         \begin{tikzpicture}[scale=2]
%              \board[-tlbr]{1}{1}
%              \row{0}{1}
%         \end{tikzpicture}
%         \caption{\TransFigureZoomedCellCaption}
%     \end{figure}
% \end{multicols}
%
% \subsection{\TransSubsectionColorScheme}
%
% \TransSubsectionColorSchemeContent
%
% \def\ColorTagZero{\textcolor{color0}{|color0|}}
% \def\ColorTagOne{\textcolor{color1}{|color1|}}
% \def\ColorTagTwo{\textcolor{color2}{|color2|}}
% \def\ColorTagThree{\textcolor{color3}{|color3|}}
% \def\ColorTagFour{\textcolor{color4}{|color4|}}
% \def\ColorTagFive{\textcolor{color5}{|color5|}}
% \def\ColorTagSix{\textcolor{color6}{|color6|}}
% \def\ColorTagSeven{\textcolor{color7}{|color7|}}
% \def\ColorTagEight{\textcolor{color8}{|color8|}}
% \def\ColorTagBorder{\textcolor{cborder}{|cborder|}}
% \def\ColorTagLight{\colorbox{cshade}{\textcolor{clight}{|clight|}}}
% \def\ColorTagShade{\textcolor{cshade}{|cshade|}}
% \def\ColorTagFail{\textcolor{cfail}{|cfail|}}
%
% \begin{table}[H]
%     \centering
%     \TransSubsectionColorSchemeTableContent
%     \caption{\TransSubsectionColorSchemeTableCaption}
% \end{table}
%
% \subsection{\TransSubsectionCellElements}
%
% \begin{function}[added=2022-01-30]{\flag}
%   \begin{syntax}
%      \tn{flag}
%   \end{syntax}
% \qquad \TransCommandFlagContent
% \end{function}
%
% \begin{figure}[!htp]
%     \centering
%     \begin{tikzpicture}
%         \begin{scope}[scale=10]
%             \flag
%             \draw[gray] (0,0) rectangle (16, -16);
%             \draw[dashed] (0,-4) node[left] {-4} -- (8.5,-4);
%             \draw[dashed,red] (16,-4.5) node[right] {-4.5} -- (8.5,-4.5);
%             \draw[dashed,red] (0,-6.5) node[left] {-6.5} -- (4,-6.5);
%             \draw[dashed,red] (16,-8.5) node[right] {-8.5} -- (8.5,-8.5);
%             \draw[dashed] (0,-11) node[left] {-11} -- (6.5,-11);
%             \draw[dashed] (16,-12) node[right] {-12} -- (12,-12);
%             \draw[dashed] (0,-13) node[left] {-13} -- (5,-13);
%             \draw[dashed,red] (4,0) node[above] {4} -- (4,-6.5);
%             \draw[dashed] (5,-16) node[below] {5} -- (5,-13);
%             \draw[dashed] (6.5,0) node[above] {6.5} -- (6.5,-11);
%             \draw[dashed] (8.5,-16) node[below] {8.5} -- (8.5,-13);
%             \draw[dashed] (9.5,0) node[above] {9.5} -- (9.5,-4);
%             \draw[dashed] (10.5,-16) node[below] {10.5} -- (10.5,-13);
%             \draw[dashed] (12,0) node[above] {12} -- (12,-12);
%         \end{scope}
%     \end{tikzpicture} \quad
%     \begin{tikzpicture}
%         \begin{scope}[scale=12]
%             \mine
%         \end{scope}
%     \end{tikzpicture}
%     \caption{\TransFigureFlagMineCaption}
% \end{figure}
%
% \begin{function}[added=2022-01-30]{\mine}
%   \begin{syntax}
%      \tn{mine}
%   \end{syntax}
% \qquad \TransCommandMineContent
% \end{function}
%
% \begin{function}[added=2022-01-30]{\cellup}
%   \begin{syntax}
%      \tn{cellup}
%   \end{syntax}
% \qquad \TransCommandCellUpContent
% \end{function}
%
% \begin{function}[added=2022-01-30]{\celldown}
%   \begin{syntax}
%      \tn{celldown}
%   \end{syntax}
% \qquad \TransCommandCellDownContent
% \end{function}
%
% \begin{figure}[!htp]
%     \centering
%     \begin{tikzpicture}[scale=6]
%          \cellup
%          \draw[thick, cborder] (0,0) rectangle (16, -16);
%          \fill[red] (8, -8) circle (0.2);
%     \end{tikzpicture}
%     \quad
%     \begin{tikzpicture}[scale=6]
%          \celldown
%          \draw[thick, cborder] (0,0) rectangle (16, -16);
%          \fill[red] (8.5, -8.5) circle (0.2);
%     \end{tikzpicture}
%     \caption{\TransFigureCellUpDownCaption}
% \end{figure}
%
% \begin{function}[added=2022-01-30, updated=2022-11-23]{\cellnum}
%   \begin{syntax}
%      \tn{cellnum} \marg{num=1|2|3|4|5|6|7|8}
%   \end{syntax}
% \qquad \TransCommandCellNumContent
% \end{function}
%
% \subsection{\TransSubsectionFrameElements}
%
% \begin{function}[added=2022-01-30, updated=2022-11-23]{\border}
%   \begin{syntax}
%      \tn{border} \oarg{border\_type=-|t|l|b|r}
%   \end{syntax}
% \qquad \TransCommandBorderContent
% \end{function}
%
% \begin{figure}[!htp]
%     \centering
%     \begin{tikzpicture}
%         \begin{scope}[scale=4]
%             \border
%             \draw[dashed] (-14,0) -- (30,0);
%             \draw[dashed] (-14,-16) -- (30,-16);
%             \draw[dashed] (0,14) -- (0,-30);
%             \draw[dashed] (16,14) -- (16,-30);
%             \node at (8, 6) {t};
%             \node at (22, -8) {r};
%             \node at (8, -22) {b};
%             \node at (-6, -8) {l};
%         \end{scope}
%     \end{tikzpicture}
%     \caption{\TransFigureBorderCaption}
% \end{figure}
%
% \subsection{\TransSubsectionModules}
%
% \TransSubsectionModulesContent
%
% \begin{function}[added=2022-01-30, updated=2022-11-23]{\cell}
%   \begin{syntax}
%      \tn{cell} \marg{r} \marg{c} \marg{info}
%   \end{syntax}
% \qquad \TransCommandCellContent
% \end{function}
%
% \begin{function}[added=2022-01-30]{\row}
%   \begin{syntax}
%      \tn{row} \marg{r} \marg{seq:info}
%   \end{syntax}
% \qquad 在第 \meta{r} 行从左到右绘制多个方格. \meta{seq:info} 为一个含有 \meta{info} 的序列, 序列中的每个元素作为 |\cell| 的第三个参数. 如果希望输入多个字符, 需要使用大括号, 例如 |1{23}4| 的序列长度为3, 该序列中的元素分别为 1, 23 和 4.
% \end{function}
%
% \begin{function}[added=2022-01-30]{\col}
%   \begin{syntax}
%      \tn{col} \marg{c} \marg{seq:info}
%   \end{syntax}
% \qquad 在第 \meta{c} 列从上到下绘制多个方格, 语法与 |\row| 一致.
% \end{function}
%
% \begin{function}[added=2022-01-30, updated=2022-11-23]{\board}
%   \begin{syntax}
%      \tn{board} \oarg{border\_type=-|t|l|b|r|x} \marg{r} \marg{c}
%   \end{syntax}
% \qquad 绘制 \meta{r} 行 \meta{c} 列的边框, t, r, b, l 为与 |\border| 对应可叠加的边框开关, 默认全部开启, 即地图中有完整边框, 如果使用了 - 标志, 代表去除某个特定边框. 此选项可以根据自己需求进行定制. 解析器只会解析最后一个 - 标志的位置, 即 |\border[-br-tl-lb-b]| 会被解析为 |\border[-b]|. 解析器不会解析重复的参数, 即 |\border[ttblbtlb]| 会被解析为 |\border[tlb]|. 可选指令 x 表示是否显示横纵坐标, 默认为不显示.
%
% 在绘制边框的过程中, 该指令会根据边框类型, 利用 \pkg{tikz} 宏包的截取指令, 自动以 \ColorTagBorder{} 为颜色, 绘制一个\textbf{0.2pt}的外边框. 同时, 为了避免格子的同色邻接区域导致的渲染问题, 该指令会提前绘制同色邻接区域.
% \end{function}
%
% \begin{figure}[!htp]
%     \captionsetup[subfloat]{labelformat=empty}
%     \centering
%     \subfloat[\texttt{\textbackslash board}]{\begin{tikzpicture}[scale=2.5]\board{1}{1}\end{tikzpicture}}
%     \quad
%     \subfloat[\texttt{\textbackslash board[tlb]} 或 \\ \texttt{\textbackslash board[-r]} ]{\begin{tikzpicture}[scale=2.5]\border[-r]\end{tikzpicture}}
%     \quad
%     \subfloat[\texttt{\textbackslash board[tlr]} 或 \\ \texttt{\textbackslash board[-b]} ]{\begin{tikzpicture}[scale=2.5]\border[-b]\end{tikzpicture}}
%     \quad
%     \subfloat[\texttt{\textbackslash board[tl]}]{\begin{tikzpicture}[scale=2.5]\border[tl]\end{tikzpicture}}
%     \caption{边框指令示例}
% \end{figure}
%
% \begin{figure}[!htp]
%     \captionsetup[subfloat]{labelformat=empty}
%     \centering
%     \subfloat[\texttt{\textbackslash board[-x]}]{\begin{tikzpicture}\board[-x]{3}{3}\end{tikzpicture}}
%     \quad
%     \subfloat[\texttt{\textbackslash board[tlx]}]{\begin{tikzpicture}\board[tlx]{3}{3}\end{tikzpicture}}
%     \quad
%     \subfloat[\texttt{\textbackslash board[-tlx]}]{\begin{tikzpicture}\board[-tlx]{3}{3}\end{tikzpicture}}
%     \caption{横纵坐标显示示例}
% \end{figure}
%
% \begin{function}[added=2022-01-30]{\colorcell}
%   \begin{syntax}
%      \tn{colorcell} \marg{color} \marg{seq: pos}
%   \end{syntax}
% \qquad 将\meta{seq: pos}位置的方格染色为\meta{color}颜色. \meta{color} 仅能传入 |\cell| 中接受的颜色类型, 但是与|\cell|仅支持单一颜色不同的是, \meta{color}支持填入至多 4 种不同颜色, 颜色将从方格部分逆时针填充. \meta{seq: pos} 定义了一个位置序列, 格式如下:
% \begin{itemize}
%     \item 序列中的每个元素以分号(;)分隔.
%     \item 对于每个元素, 以逗号(,)分隔行区间和列区间.
%     \item 对于每个区间, 如果它是\textbf{单字符}, 则表示区间的起始位置和终止位置相同, 均为该字符; 否则需要以减号(-)分隔起始位置和终止位置.
% \end{itemize}
% 下面给出一些示例:
%  \begin{ctexexam}
%    \colorcell{g}{0,0-1;0-1,3} % 代表用绿色染色0行, 0-1列和0-1行, 3列的4个格子.
%    \colorcell{r}{0,4;1-2,0-2} % 代表用红色染色0行, 4列和1-2行, 0-2列的7个格子.
%    \colorcell{rg}{3,0-2} % 代表用红色和绿色染色3行, 0-2列的3个格子.
%    \colorcell{rgb}{4,0-2} % 代表用红色, 绿色和蓝色染色4行, 0-2列的3个格子.
%    \colorcell{rgby}{3-4,3-4} % 代表用红色, 绿色, 蓝色和黄色染色3-4行, 3-4列的4个格子.
%  \end{ctexexam}
% \end{function}
%
% \begin{multicols}{2}
% \iffalse
%<*internal>
% \fi
\begin{verbatim}
    \begin{tikzpicture}
        \board{3}{5}
        \row{0}{f{20}123}
        \row{1}{A-405}
        \row{2}{m-678}
        \colorcell{g}{0,0-1;0-1,3}
        \colorcell{r}{0,4;1-2,0-2}
        \colorcell{rg}{3,0-2}
        \colorcell{rgb}{4,0-2}
        \colorcell{rgby}{3-4,3-4}
    \end{tikzpicture}
\end{verbatim}
% \iffalse
%</internal>
% \fi
%     \columnbreak
%     \begin{figure}[H]
%         \centering
%         \begin{tikzpicture}
%             \board{5}{5}
%             \row{0}{f{20}123}
%             \row{1}{A-405}
%             \row{2}{m-678}
%             \row{3}{-----}
%             \row{4}{-----}
%             \colorcell{g}{0,0-1;0-1,3}
%             \colorcell{r}{0,4;1-2,0-2}
%             \colorcell{rg}{3,0-2}
%             \colorcell{rgb}{4,0-2}
%             \colorcell{rgby}{3-4,3-4}
%         \end{tikzpicture}
%         \caption{方格染色示例}
%     \end{figure}
% \end{multicols}
%
% \StopEventually{}
%
% \begin{implementation}
%
% \clearpage
%
% \section{代码实现}
% 定义 \pkg{tikz-minesweeper} 宏包的依赖, 分别为 \LaTeXiii{} 和 \pkg{tikz} 宏包.
%    \begin{macrocode}

\RequirePackage{expl3}
\RequirePackage{tikz}

%    \end{macrocode}
%
% 定义 \pkg{tikz-minesweeper} 宏包的配色方案. 在实际使用该宏包时, 可以使用类似方法定义自己的配色方案.
%    \begin{macrocode}
\definecolor{color0}{RGB}{192,192,192} % color for blank cell
\definecolor{color1}{RGB}{0,0,255} % color for cell 1
\definecolor{color2}{RGB}{0,128,0} % color for cell 2
\definecolor{color3}{RGB}{255,0,0} % color for cell 3
\definecolor{color4}{RGB}{0,0,128} % color for cell 4
\definecolor{color5}{RGB}{122,43,26} % color for cell 5
\definecolor{color6}{RGB}{0,128,128} % color for cell 6
\definecolor{color7}{RGB}{0,0,0} % color for cell 7
\definecolor{color8}{RGB}{128,128,128} % color for cell 8
\definecolor{cborder}{RGB}{160,160,160} % color for border
\definecolor{clight}{RGB}{255,255,255} % color for whitespace
\definecolor{cshade}{RGB}{128,128,128} % color for shades
\definecolor{cfail}{RGB}{255,0,0} % color for failed cells

%    \end{macrocode}
%
% 定义 \pkg{tikz-minesweeper} 宏包的基本单位, 即\textbf{1pt}.
%    \begin{macrocode}
\tikzset{x=1pt, y=1pt} % Default unit

%    \end{macrocode}
%
% 打开\LaTeXiii{}编程环境.
%    \begin{macrocode}
\ExplSyntaxOn

%    \end{macrocode}
%
% 定义 \pkg{tikz-minesweeper} 宏包中的变量.
%    \begin{macrocode}
\str_const:Nn \c__msweeper_valid_numbers_str {012345678}
\str_const:Nn \c__msweeper_valid_colors_str {rgbcyov}
\str_new:N \l__msweeper_cellcolor_str
\str_new:N \l__msweeper_qtrcellcolor_str
\int_new:N \l__msweeper_rows_int
\int_new:N \l__msweeper_cols_int
\int_new:N \l__msweeper_tmargin_int
\int_new:N \l__msweeper_lmargin_int
\int_new:N \l__msweeper_bmargin_int
\int_new:N \l__msweeper_rmargin_int

%    \end{macrocode}
%
% \begin{macro}{\msweeper_flag:}
% \qquad 绘制旗子.
%    \begin{macrocode}
\cs_set:Nn \msweeper_flag: {
    \fill[color7] (5, -13) rectangle (12, -12); % lower base
    \fill[color7] (6.5, -13) rectangle (10.5, -11); % upper base
    \fill[color7] (8.5, -13) rectangle (9.5, -4); % pole
    \fill[color3] (8.5, -4.5) -- (4, -6.5) -- (8.5, -8.5) -- cycle; % red flag
}

%    \end{macrocode}
% \end{macro}
%
% \begin{macro}{\msweeper_mine:}
% \qquad 绘制地雷.
%    \begin{macrocode}
\cs_set:Nn \msweeper_mine: {
    \int_step_inline:nnnn {0} {45} {135} {
        \fill[color7, xshift=8.5, yshift=-8.5, rotate=##1] (-6.5, 0.4) -- (0, 1.5) --
            (6.5, 0.4) -- (6.5, -0.4) -- (0, -1.5) -- (-6.5, -0.4) -- cycle; % spikes
    }
    \shade[ball~color=color7] (8.5,-8.5) circle (4.5); % body
}

%    \end{macrocode}
% \end{macro}
%
% \begin{macro}{\msweeper_mine_shaded:}
% \qquad 绘制半透明的地雷. 需要注意的是, 半透明的雷的颜色没有与其它颜色解耦, 如果定义了不同的配色方案, 请谨慎绘制这种地雷.
%    \begin{macrocode}
\cs_set:Nn \msweeper_mine_shaded: {
    \int_step_inline:nnnn {0} {45} {135} {
        \fill[cshade, xshift=8.5, yshift=-8.5, rotate=##1] (-6.5, 0.4) -- (0, 1.5) --
            (6.5, 0.4) -- (6.5, -0.4) -- (0, -1.5) -- (-6.5, -0.4) -- cycle; % spikes
    }
    \shade[ball~color=cshade] (8.5,-8.5) circle (4.5); % body
}

%    \end{macrocode}
% \end{macro}
%
% \begin{macro}{\msweeper_cellup:}
% \qquad 绘制鼠标弹起时的方格.
%    \begin{macrocode}
\cs_set:Nn \msweeper_cellup: {
    \fill[clight] (0, 0) -- (16, 0) -- (0, -16) -- cycle; % highlight
    \fill[cshade] (16, -16) -- (0, -16) -- (16, 0) -- cycle; % shade
    \fill[color0] (2, -2) rectangle (14, -14); % background
}

%    \end{macrocode}
% \end{macro}
%
% \begin{macro}{\msweeper_celldown:}
% \qquad 绘制鼠标按下后的方格.
%    \begin{macrocode}
\cs_set:Nn \msweeper_celldown: {
    \fill[cshade] (0, -16) rectangle (16, 0); % border
    \fill[color0] (1, -16) rectangle (16, -1); % background
}

%    \end{macrocode}
% \end{macro}
%
% \begin{macro}{\msweeper_cellfail:}
% \qquad 绘制由于踩雷而失败的方格.
%    \begin{macrocode}
\cs_set:Nn \msweeper_cellfail: {
    \fill[cshade] (0, -16) rectangle (16, 0); % border
    \fill[cfail] (1, -16) rectangle (16, -1); % background
}

%    \end{macrocode}
% \end{macro}
%
% \begin{macro}{\msweeper_misflagfail:}
% \qquad 绘制由于标记错误而失败的方格.
%    \begin{macrocode}
\cs_set:Nn \msweeper_misflagfail: {
    \msweeper_mine:
    \draw[cfail, line~width=1] (2.5, -2.5) -- (14.5, -14.5); % cross line
    \draw[cfail, line~width=1] (2.5, -14.5) -- (14.5, -2.5); % cross line
}

%    \end{macrocode}
% \end{macro}
%
% \begin{macro}{\msweeper_cellnum:n}
% \qquad 用 \ColorTagOne{} 至 \ColorTagEight{} 绘制给定的矢量化数字 1 至 8, 接受一个参数 |#1|. 参数 |#1| 仅支持传入 1 至 8, 传入其它值将会被忽略.
%    \begin{macrocode}
\cs_set:Npn \msweeper_cellnum:n #1 {
    \str_case:nn {#1} {
        % Textures come from https://github.com/Minesweeper-World/MS-Texture/
        % Positions tweaked by ./support/tikz-path-xyshift.py
        {1} { \fill[color1] (5, -11.5) -- (5, -13.5) -- (12, -13.5) -- (12, -11.5) --
            (10, -11.5) -- (10, -3.5) -- (8.5, -3.5) -- (5, -7.0) -- (5, -7.5) --
            (7, -7.5) -- (7, -11.5) -- cycle; }
        {2} { \fill[color2] (3.5, -6.5) .. controls (3.5, -3.5) .. (7.0, -3.5) --
            (10.0, -3.5) .. controls (13.5, -3.5) .. (13.5, -6.5) .. controls
            (13.5, -9.0) and (7.5, -10.5) .. (6.5, -11.5) -- (13.5, -11.5) --
            (13.5, -13.5) -- (3.5, -13.5) -- (3.5, -12.0) .. controls (3.5, -8.5)
            and (10.5, -8.5) .. (10.5, -6.5) .. controls (10.5, -5.5) ..
            (9.5, -5.5) -- (7.5, -5.5) .. controls (6.5, -5.5) ..
            (6.5, -6.5) -- cycle; }
        {3} { \fill[color3] (3.5, -3.5) -- (10.0, -3.5) .. controls (13.5, -3.5) ..
            (13.5, -6.5) .. controls (13.5, -8.0) .. (12.0, -8.5) .. controls
            (13.5, -9.0) .. (13.5, -10.5) .. controls (13.5, -13.5) ..
            (10.0, -13.5) -- (3.5, -13.5) -- (3.5, -11.5) -- (9.5, -11.5) ..
            controls (10.5, -11.5) .. (10.5, -10.5) .. controls (10.5, -9.5) ..
            (9.5, -9.5) -- (6.5, -9.5) -- (6.5, -7.5) -- (9.5, -7.5) .. controls
            (10.5, -7.5) .. (10.5, -6.5) .. controls (10.5, -5.5) .. (9.5, -5.5) --
            (3.5, -5.5) -- cycle; }
        {4} { \fill[color4] (6.0, -3.5) -- (3.5, -8.5) -- (3.5, -9.5) --
            (9.5, -9.5) -- (9.5, -13.5) -- (12.5, -13.5) -- (12.5, -9.5) --
            (13.5, -9.5) -- (13.5, -7.5) -- (12.5, -7.5) -- (12.5, -3.5) --
            (9.5, -3.5) -- (9.5, -7.5) -- (7.0, -7.5) -- (9.0, -3.5) -- cycle; }
        {5} { \fill[color5] (3.5, -3.5) -- (13.5, -3.5) -- (13.5, -5.5) --
            (6.5, -5.5) -- (6.5, -7.5) -- (10.0, -7.5) .. controls (13.5, -7.5) ..
            (13.5, -10.5) .. controls (13.5, -13.5) .. (10.0, -13.5) --
            (3.5, -13.5) -- (3.5, -11.5) -- (9.5, -11.5) .. controls
            (10.5, -11.5) .. (10.5, -10.5) .. controls (10.5, -9.5) ..
            (9.5, -9.5) -- (3.5, -9.5) -- cycle; }
        {6} { \fill[color6] (12.5, -3.5) -- (12.5, -5.5) -- (8.5, -5.5) ..
            controls (6.5, -5.5) .. (6.5, -6.5) -- (6.5, -10.5) .. controls
            (6.5, -11.5) .. (8.5, -11.5) .. controls (10.5, -11.5) ..
            (10.5, -10.5) .. controls (10.5, -9.5) .. (8.5, -9.5) -- (6.5, -9.5) --
            (6.5, -7.5) -- (10.0, -7.5) .. controls (13.5, -7.5) ..
            (13.5, -10.5) .. controls (13.5, -13.5) .. (10.0, -13.5) --
            (7.0, -13.5) .. controls (3.5, -13.5) .. (3.5, -10.5) --
            (3.5, -6.5) .. controls (3.5, -3.5) .. (7.0, -3.5) -- cycle; }
        {7} { \fill[color7] (3.5, -3.5) -- (13.5, -3.5) -- (13.5, -6.5) --
            (10.0, -13.5) -- (7.0, -13.5) -- (10.5, -6.5) -- (10.5, -5.5) --
            (3.5, -5.5) -- cycle; }
        {8} { \fill[color8] (7.0, -3.5) -- (10.0, -3.5) .. controls (13.5, -3.5) ..
            (13.5, -6.5) .. controls (13.5, -8.0) .. (12.0, -8.5) -- (10.5, -8.5) --
            (10.5, -6.5) .. controls (10.5, -5.5) .. (9.5, -5.5) -- (7.5, -5.5) ..
            controls (6.5, -5.5) .. (6.5, -6.5) .. controls (6.5, -7.5) ..
            (7.5, -7.5) -- (9.5, -7.5) .. controls (10.5, -7.5) .. (10.5, -6.5) --
            (10.5, -10.5) .. controls (10.5, -9.5) .. (9.5, -9.5) -- (7.5, -9.5) ..
            controls (6.5, -9.5) .. (6.5, -10.5) .. controls (6.5, -11.5) ..
            (7.5, -11.5) -- (9.5, -11.5) .. controls (10.5, -11.5) ..
            (10.5, -10.5) -- (10.5, -8.5) -- (12.0, -8.5) .. controls
            (13.5, -9.0) .. (13.5, -10.5) .. controls (13.5, -13.5) ..
            (10.0, -13.5) -- (7.0, -13.5) .. controls (3.5, -13.5) .. (3.5, -10.5) ..
            controls (3.5, -9.0) .. (5.0, -8.5) .. controls (3.5, -8.0) ..
            (3.5, -6.5) .. controls (3.5, -3.5) .. (7.0, -3.5) -- cycle; }
    }
}

%    \end{macrocode}
% \end{macro}
%
% \begin{macro}{\msweeper_cellcolored:n}
% \qquad 绘制鼠标按下时的染色方格, 接受一个参数 |#1|. 参数 |#1| 仅支持传入 r, g, b, c, y, o, v 中的一个, 传入其它值将会被忽略. 参数 |#1| 的最长长度为 4, 如果传入的参数长度小于 4, 则会在参数的左/右侧补充与其相同的字符, 如果传入的参数长度大于 4, 则会直接被截断为 4 个字符.
%    \begin{macrocode}
\cs_set:Npn \msweeper_cellcolored:n #1 {
    \str_set:Nn \l__msweeper_cellcolor_str {#1}
    \int_case:nn { \str_count:n {#1} } {
        {1} { \str_set:Nn \l__msweeper_cellcolor_str {#1#1#1#1} }
        {2} {
            \str_put_left:Nx \l__msweeper_cellcolor_str { \str_head:n {#1} }
            \str_put_right:Nx \l__msweeper_cellcolor_str { \str_tail:n {#1} }
        }
        {3} { \str_put_left:Nx \l__msweeper_cellcolor_str { \str_head:n {#1} } }
    }

    \int_step_inline:nnn {0} {3} {
        \str_case_e:nnT { \str_item:Nn \l__msweeper_cellcolor_str {##1 + 1} } {
            {r} { \str_set:Nn \l__msweeper_qtrcellcolor_str {red} }
            {g} { \str_set:Nn \l__msweeper_qtrcellcolor_str {green} }
            {b} { \str_set:Nn \l__msweeper_qtrcellcolor_str {blue} }
            {c} { \str_set:Nn \l__msweeper_qtrcellcolor_str {cyan} }
            {y} { \str_set:Nn \l__msweeper_qtrcellcolor_str {yellow} }
            {o} { \str_set:Nn \l__msweeper_qtrcellcolor_str {orange} }
            {v} { \str_set:Nn \l__msweeper_qtrcellcolor_str {violet} }
        } {
            \begin{scope}[xshift=8, yshift=-8]
                \fill[color=\l__msweeper_qtrcellcolor_str, opacity=0.2, rotate=90*##1]
                    (0, 0) -- (8, 8) -- (-8, 8) -- cycle;
            \end{scope}
        }
    }
}

%    \end{macrocode}
% \end{macro}
%
% \begin{macro}{\msweeper_cell:nnn}
% \qquad 绘制方格, 接受三个参数 |#1|, |#2|, |#3|. 在坐标(|#3|, |#2|)处绘制一个方格, 方格中的内容由 |#1| 决定, |#1|可以传入任意参数. 该函数依次检查 |#1| 是否为数字参数 (012345678), 颜色参数 (rgbcyov), 雷与旗帜参数 (fmsne-), 除上述字符外的单字符参数以及多字符参数.
%    \begin{macrocode}
\cs_set:Npn \msweeper_cell:nnn #1#2#3 {
    \sffamily
    \begin{scope}[xshift=#3*16, yshift=-#2*16]
        \tl_if_single:nTF {#1} {
            % single token
            \str_if_in:NnTF \c__msweeper_valid_numbers_str {#1} {
                \msweeper_celldown: \msweeper_cellnum:n {#1} % number
            } {
                \str_if_in:NnTF \c__msweeper_valid_colors_str {#1} {
                    \msweeper_cellup: \msweeper_cellcolored:n {#1} % color
                } {
                    \str_case:nnF {#1} {
                        {f} { \msweeper_cellup: \msweeper_flag: } % flag
                        {m} { \msweeper_celldown: \msweeper_mine: } % mine
                        {s} { \msweeper_celldown: \msweeper_mine_shaded: } % shaded
                        {n} { \msweeper_cellfail: \msweeper_mine: } % blasted
                        {e} { \msweeper_celldown: \msweeper_misflagfail: } % misflagged
                        {-} { \msweeper_cellup: } % empty
                    } {
                        \msweeper_cellup:
                        \node[text~centered, text={color7}] at (8, -8) {\textbf{#1}};
                    } % normal label
                }
            }
        } {
            \msweeper_cellup:
            \node[text~centered, text={color7}] at (8, -8) {\textbf{\tiny #1}};
        } % tiny label
    \end{scope}
}

%    \end{macrocode}
% \end{macro}
%
% \begin{macro}{\msweeper_line:nnn}
% \qquad 绘制一行或者一列的格子, 接受三个参数 |#1|, |#2|, |#3|. |#1| 为行号或者列号, |#2| 为格子内容, |#3| 为按行首(r)或者按列首(c)依次绘制格子, 其余参数将不会绘制.
%    \begin{macrocode}
\cs_set:Npn \msweeper_line:nnn #1#2#3 {
    \int_zero:N \l_tmpa_int
    \tl_map_inline:nn {#2} {
        \str_case:nn {#3} {
            {r} { \msweeper_cell:nnn {##1}{#1}{\l_tmpa_int} }
            {c} { \msweeper_cell:nnn {##1}{\l_tmpa_int}{#1} }
        }
        \int_incr:N \l_tmpa_int
    }
}

%    \end{macrocode}
% \end{macro}
%
% \begin{macro}{\msweeper_row_col_marker:}
% \qquad 绘制行列坐标数字.
%    \begin{macrocode}
\cs_set:Nn \msweeper_row_col_marker: {
    \sffamily
    % show axis x
    \int_step_inline:nnn {0} {\l__msweeper_cols_int - 1} {
        \node[text~centered, text={color7}, xshift=16*##1] at
            (8, 6 + \l__msweeper_tmargin_int) {\textbf{\tiny ##1}};
    }

    % show axis y
    \int_step_inline:nnn {0} {\l__msweeper_rows_int - 1} {
        \node[text~centered, text={color7}, yshift=-16*##1] at
            (-6 - \l__msweeper_lmargin_int, -8) {\textbf{\tiny ##1}};
    }
}

%    \end{macrocode}
% \end{macro}
%
% \begin{macro}{\msweeper_raw_border:}
% \qquad 绘制原始的边框, 用户在绘制盘面时传入的边框参数将会影响最终边框的形状.
%    \begin{macrocode}
\cs_set:Nn \msweeper_raw_border: {
    \fill[clight] (-12, 12) -- (\l__msweeper_cols_int*16 + 12, 12) --
        (\l__msweeper_cols_int*16, 0) -- (0, -\l__msweeper_rows_int*16) --
        (-12, -\l__msweeper_rows_int*16 - 12) -- cycle;
    \fill[cborder] (\l__msweeper_cols_int*16 + 12, -\l__msweeper_rows_int*16 - 12) --
        (-12, -\l__msweeper_rows_int*16 - 12) -- (0, -\l__msweeper_rows_int*16) --
        (\l__msweeper_cols_int*16, 0) -- (\l__msweeper_cols_int*16 + 12, 12) -- cycle;
    \fill[color0] (-9, 9) rectangle
        (\l__msweeper_cols_int*16 + 9, -\l__msweeper_rows_int*16 - 9);
    \fill[cborder] (-3, 3) -- (\l__msweeper_cols_int*16 + 3, 3) --
        (\l__msweeper_cols_int*16, 0) -- (0, -\l__msweeper_rows_int*16) --
        (-3, -\l__msweeper_rows_int*16 - 3) -- cycle;
    \fill[clight] (\l__msweeper_cols_int*16 + 3, -\l__msweeper_rows_int*16 - 3) --
        (-3, -\l__msweeper_rows_int*16 - 3) -- (0, -\l__msweeper_rows_int*16) --
        (\l__msweeper_cols_int*16, 0) -- (\l__msweeper_cols_int*16 + 3, 3) -- cycle;
    \fill[clight] (0, 0) rectangle
        (\l__msweeper_cols_int*16, -\l__msweeper_rows_int*16);
}

%    \end{macrocode}
% \end{macro}
%
% \begin{macro}{\msweeper_gap_filler:}
% \qquad 由于渲染问题, 绘制格子时, 同色区域之间可能会出现一个空白的缝隙, 该函数用于提前填充这些缝隙.
%    \begin{macrocode}
\cs_set:Nn \msweeper_gap_filler: {
    % draw lines in advance to avoid gaps
    \int_step_variable:nnNn {1} {\l__msweeper_cols_int - 1} \l_tmpa_tl {
        \int_step_variable:nnNn {1} {\l__msweeper_rows_int} \l_tmpb_tl {
            \fill[cshade, xshift=\l_tmpa_tl*16, yshift=-\l_tmpb_tl * 16] (0, 0) --
                (-0.5, 0.5) -- (-0.5, 14.5) -- (0, 15) -- (-0.5, 15.5) -- (0, 16) --
                (0.5, 15.5) -- (0, 15) -- (0.5, 14.5) -- (0.5, 0.5) -- cycle;
        }
    }

    \int_step_variable:nnNn {1} {\l__msweeper_rows_int - 1} \l_tmpb_tl {
        \int_step_variable:nnNn {1} {\l__msweeper_cols_int} \l_tmpa_tl {
            \fill[cshade, xshift=\l_tmpa_tl*16, yshift=-\l_tmpb_tl * 16] (0, 0) --
                (-0.5, -0.5) -- (-14.5, -0.5) -- (-15, 0) -- (-15.5, -0.5) --
                (-16, 0) -- (-15.5, 0.5) -- (-15, 0) -- (-14.5, 0.5) --
                (-0.5, 0.5) -- cycle;
        }
    }
}

%    \end{macrocode}
% \end{macro}
%
% \begin{macro}{\msweeper_border:nnn}
% \qquad 绘制边框, 接受三个参数 |#1|, |#2|, |#3|. |#1| 为边框种类参数, |#2| 为行数, |#3| 为列数. 参数 |#1| 可以传入 t, l, b, r, -, x 中的任意多个值.
%    \begin{macrocode}
\cs_set:Npn \msweeper_border:nnn #1#2#3 {
    \int_set:Nn \l__msweeper_rows_int {#2}
    \int_set:Nn \l__msweeper_cols_int {#3}

    \bool_set_false:N \l_tmpa_bool
    \int_zero:N \l__msweeper_tmargin_int
    \int_zero:N \l__msweeper_lmargin_int
    \int_zero:N \l__msweeper_bmargin_int
    \int_zero:N \l__msweeper_rmargin_int

    \tl_map_inline:nn {#1} {
        \str_case:nn {##1} {
            {-} {
                \bool_set_true:N \l_tmpa_bool
                \int_set:Nn \l__msweeper_tmargin_int {12}
                \int_set:Nn \l__msweeper_lmargin_int {12}
                \int_set:Nn \l__msweeper_bmargin_int {12}
                \int_set:Nn \l__msweeper_rmargin_int {12}
            }
            {t} { \int_set:Nn \l__msweeper_tmargin_int {12 - 12 * \l_tmpa_bool} }
            {l} { \int_set:Nn \l__msweeper_lmargin_int {12 - 12 * \l_tmpa_bool} }
            {b} { \int_set:Nn \l__msweeper_bmargin_int {12 - 12 * \l_tmpa_bool} }
            {r} { \int_set:Nn \l__msweeper_rmargin_int {12 - 12 * \l_tmpa_bool} }
        }
    }

    % draw row and column numbers
    \str_if_in:nnT {#1} {x} { \msweeper_row_col_marker: }

    % draw outer border with clipping
    % clip the area inside of a scope to avoid side effects
    \begin{scope}
        \clip[preaction={draw, color=cborder, line~width=0.8}]
            (-\l__msweeper_lmargin_int, \l__msweeper_tmargin_int) rectangle
            (\l__msweeper_cols_int*16 + \l__msweeper_rmargin_int,
            -\l__msweeper_rows_int*16 - \l__msweeper_bmargin_int);
        \msweeper_raw_border:
    \end{scope}

    \msweeper_gap_filler:
}

%    \end{macrocode}
% \end{macro}
%
% \begin{macro}{\msweeper_colorcell:nn}
% \qquad 给多个区域的方格染色, 接受两个参数 |#1|, |#2|. |#1| 仅能传入 |\msweeper_cellcolored:n| 中接受的颜色, |#2| 传入待绘制的多个区域.
%    \begin{macrocode}
\cs_set:Npn \msweeper_colorcell:nn #1#2 {
    \tl_set:Nn \l_tmpa_tl {#2}
    \seq_set_split:NnV \l_tmpa_seq {;} \l_tmpa_tl
    \seq_map_inline:Nn \l_tmpa_seq {
        \tl_set:Nx \l_tmpb_tl {##1}
        \seq_set_split:NnV \l_tmpb_seq {,} \l_tmpb_tl
        \tl_set:Nx \l__ms_width_tl {\seq_item:Nn \l_tmpb_seq {1}}
        \tl_set:Nx \l__ms_height_tl {\seq_item:Nn \l_tmpb_seq {-1}}
        \seq_set_split:NnV \l__ms_width_seq {-} \l__ms_width_tl
        \seq_set_split:NnV \l__ms_height_seq {-} \l__ms_height_tl
        \tl_set:Nx \l__ms_widthx_tl {\seq_item:Nn \l__ms_width_seq {1}}
        \tl_set:Nx \l__ms_widthy_tl {\seq_item:Nn \l__ms_width_seq {-1}}
        \tl_set:Nx \l__ms_heightx_tl {\seq_item:Nn \l__ms_height_seq {1}}
        \tl_set:Nx \l__ms_heighty_tl {\seq_item:Nn \l__ms_height_seq {-1}}
        \int_step_variable:nnNn {\l__ms_widthx_tl} {\l__ms_widthy_tl} \l_tmpc_tl {
            \int_step_variable:nnNn {\l__ms_heightx_tl} {\l__ms_heighty_tl} \l_tmpd_tl {
                \begin{scope}[xshift=\l_tmpd_tl*16,yshift=-\l_tmpc_tl*16]
                    \msweeper_cellcolored:n {#1}
                \end{scope}
            }
        }
    }
}

%    \end{macrocode}
% \end{macro}
%
% 提供不显式调用 \LaTeXiii{} 的用户接口.
%    \begin{macrocode}
\newcommand{\flag}{\msweeper_flag:}
\newcommand{\mine}{\msweeper_mine:}
\newcommand{\cellup}{\msweeper_cellup:}
\newcommand{\celldown}{\msweeper_celldown:}
\newcommand{\cellnum}[1]{\msweeper_cellnum:n {#1}}
\newcommand{\cell}[3]{\msweeper_cell:nnn {#1}{#2}{#3}}
\newcommand{\row}[2]{\msweeper_line:nnn {#1}{#2}{r}}
\newcommand{\col}[2]{\msweeper_line:nnn {#1}{#2}{c}}
\newcommand{\border}[1][-]{\msweeper_border:nnn {#1}{1}{1}}
\newcommand{\board}[3][-]{\msweeper_border:nnn {#1}{#2}{#3}}
\newcommand{\colorcell}[2]{\msweeper_colorcell:nn {#1}{#2}}

%    \end{macrocode}
%
% 关闭 \LaTeXiii{} 编程环境.
%    \begin{macrocode}
\ExplSyntaxOff

%    \end{macrocode}
%
% \end{implementation}
%
% \Finale
%
\endinput
