% \iffalse meta-comment
% !TeX program  = XeLaTeX
% !TeX encoding = UTF-8
% Copyright (C) 2021-2022 by Tian-Yi Pu, Fei-Yu Xiang
% and Yao-Yu Zhu
%
% This file may be distributed and/or modified under the
% conditions of the LaTeX Project Public License, either
% version 1.3 of this license or (at your option) any later
% version. The latest version of this license is in:
%
%    http://www.latex-project.org/lppl.txt
%
% and version 1.3 or later is part of all distributions of
% LaTeX version 2005/12/01 or later.
%
% \fi

% \iffalse
%<package>\NeedsTeXFormat{LaTeX2e}[2005/12/01]
%<package>\ProvidesPackage{tikz-minesweeper}
%<package>    [2022/01/30 v0.1.0 Draw a minesweeper board in LaTeX]
%<*batchfile>
\begingroup
\input ctxdocstrip.tex
\keepsilent
\usedir{tex/latex/tikz-minesweeper}

\preamble
Copyright (C) 2021-2022 by Tian-Yi Pu, Fei-Yu Xiang
and Yao-Yu Zhu

This file may be distributed and/or modified under the
conditions of the LaTeX Project Public License, either
version 1.3 of this license or (at your option) any later
version. The latest version of this license is in:

   http://www.latex-project.org/lppl.txt

and version 1.3 or later is part of all distributions of
LaTeX version 2005/12/01 or later.
\endpreamble

\postamble
This package consists of the files tikz-minesweeper.dtx,
                                   README.md,
             and the derived files tikz-minesweeper.sty,
                                   tikz-minesweeper.pdf.
\endpostamble

\generate{\file{tikz-minesweeper.sty}{\from{\jobname.dtx}{package}}}

\obeyspaces
\typeout{****************************************************}
\typeout{*                                                  *}
\typeout{* To finish the installation you have to move the  *}
\typeout{* following file into a directory searched by TeX: *}
\typeout{*                                                  *}
\typeout{*     tikz-minesweeper.sty                         *}
\typeout{*                                                  *}
\typeout{* To produce the documentation run the file        *}
\typeout{* tikz-minesweeper.dtx through LaTeX.              *}
\typeout{*                                                  *}
\typeout{* Happy TeXing (and minesweeping) !                *}
\typeout{*                                                  *}
\typeout{****************************************************}

\endgroup
%</batchfile>
%<*driver>
\documentclass{ctxdoc}
\usepackage{tikz-minesweeper}
\usepackage{verbatim}
\usepackage{booktabs}
\usepackage{makecell}
\usepackage{colortbl}
\usepackage{subfig}
\usepackage{float}
\usepackage{multicol}
\ctexset{
    section = {
        name = {,},
        number = \arabic{section},
    }
}
\EnableCrossrefs
\CodelineIndex
\RecordChanges
\begin{document}
    \DocInput{tikz-minesweeper.dtx}
    \IndexLayout
    \PrintChanges
\end{document}
%</driver>
% \fi
%
% \CheckSum{0}
%
% \CharacterTable
%  {Upper-case    \A\B\C\D\E\F\G\H\I\J\K\L\M\N\O\P\Q\R\S\T\U\V\W\X\Y\Z
%   Lower-case    \a\b\c\d\e\f\g\h\i\j\k\l\m\n\o\p\q\r\s\t\u\v\w\x\y\z
%   Digits        \0\1\2\3\4\5\6\7\8\9
%   Exclamation   \!     Double quote  \"     Hash (number) \#
%   Dollar        \$     Percent       \%     Ampersand     \&
%   Acute accent  \'     Left paren    \(     Right paren   \)
%   Asterisk      \*     Plus          \+     Comma         \,
%   Minus         \-     Point         \.     Solidus       \/
%   Colon         \:     Semicolon     \;     Less than     \<
%   Equals        \=     Greater than  \>     Question mark \?
%   Commercial at \@     Left bracket  \[     Backslash     \\
%   Right bracket \]     Circumflex    \^     Underscore    \_
%   Grave accent  \`     Left brace    \{     Vertical bar  \|
%   Right brace   \}     Tilde         \~}
%
% \changes{v0.1.0}{2022/01/30}{初始更新版本}
%
% \GetFileInfo{tikz-minesweeper.sty}
%
% \title{\texttt{tikz-minesweeper}宏包使用手册\thanks{本使用手册适用于\texttt{tikz-minesweeper}~\fileversion, 更新于~\filedate.}}
% \author{濮天羿~~~~向飞宇~~~~朱耀宇}
% \maketitle
%
% \section{绘制示例}
% 下面是一个\texttt{tikz-minesweeper}宏包的绘制示例, 包含了该宏包中最重要的几个指令.
%
% \begin{multicols}{2}
% \iffalse
%<*internal>
% \fi
\begin{verbatim}
    \usepackage{tikz-minesweeper}
    \begin{document}
        \begin{tikzpicture}
            \board{5}{5}
            \row{0}{A{20}12}
            \row{1}{f-40}
            \row{2}{mn67}
            \row{3}{rgbc}
            \row{4}{oltv}
            \col{4}{358y-}
        \end{tikzpicture}
    \end{document}
\end{verbatim}
% \iffalse
%</internal>
% \fi
%     \columnbreak
%     \begin{figure}[H]
%         \centering
%         \begin{tikzpicture}
%             \board{5}{5}
%             \row{0}{A{20}12}
%             \row{1}{f-40}
%             \row{2}{mn67}
%             \row{3}{rgbc}
%             \row{4}{oltv}
%             \col{4}{358y-}
%         \end{tikzpicture}
%         \caption*{示例: 在5x5区域内绘制数字、雷、旗帜、空格和标记}
%     \end{figure}
% \end{multicols}
% \section{绘制指令介绍}
%
% \subsection{基础设定}
% 本宏包中设定单位长度为\textbf{1pt}, 为了与标准扫雷游戏一致, 一格的边长是\textbf{16pt}. 下文如无特别说明, 所有方格的原点都位于左上角, 所有数据的单位均为\textbf{pt}.
%
% 如果需要缩放一个盘面, 推荐使用scope环境中的scale参数, 即采用如下语法:
% \begin{multicols}{2}
% \iffalse
%<*internal>
% \fi
\begin{verbatim}
    \begin{tikzpicture}[scale=2]
        \board{1}{1}
        \row{0}{1}
    \end{tikzpicture}
\end{verbatim}
% \iffalse
%</internal>
% \fi
%     \columnbreak
%     \begin{figure}[H]
%         \centering
%         \begin{tikzpicture}[scale=2]
%              \board[-tlbr]{1}{1}
%              \row{0}{1}
%         \end{tikzpicture}
%         \caption*{示例: 放大一个格子}
%     \end{figure}
% \end{multicols}
%
% \subsection{配色}
% \texttt{tikz-minesweeper}宏包定义了用于绘制扫雷数字和盘面的配色. 这些配色均给出了标签, 这些标签均可以直接在文档中使用.
%
% \begin{table}[H]
%     \centering
%     \begin{tabular}{ccccc}
%         \toprule
%         颜色标签                                    & R   & G   & B   & 场景                              \\
%         \midrule
%         \textcolor{color0}{color0}                  & 192 & 192 & 192 & 格子内部 / 外边框                 \\
%         \textcolor{color1}{color1}                  & 0   & 0   & 255 & 数字1                             \\
%         \textcolor{color2}{color2}                  & 0   & 128 & 0   & 数字2                             \\
%         \textcolor{color3}{color3}                  & 255 & 0   & 0   & 数字3                             \\
%         \textcolor{color4}{color4}                  & 0   & 0   & 128 & 数字4                             \\
%         \textcolor{color5}{color5}                  & 128 & 0   & 0   & 数字5                             \\
%         \textcolor{color6}{color6}                  & 0   & 128 & 128 & 数字6                             \\
%         \textcolor{color7}{color7}                  & 0   & 0   & 0   & 数字7                             \\
%         \textcolor{color8}{color8}                  & 128 & 128 & 128 & 数字8                             \\
%         \textcolor{cborder}{cborder}                & 160 & 160 & 160 & 边框阴影                          \\
%         \cellcolor{black}\textcolor{clight}{clight} & 255 & 255 & 255 & 高亮                              \\
%         \textcolor{cshade}{cshade}                  & 128 & 128 & 128 & 格子阴影 / 按下的格子边界           \\
%         \textcolor{cfail}{cfail}                    & 255 & 0   & 0   & 导致失败的格子                     \\
%         \bottomrule
%     \end{tabular}
%     \caption*{颜色对照表}
% \end{table}
%
% \subsection{方格元素}
%
% \DescribeMacro{\flag} 在一个方格中绘制旗子.
%
% \begin{figure}[!htp]
%     \centering
%     \begin{tikzpicture}
%         \begin{scope}[scale=10]
%             \flag
%             \draw[gray] (0,0) rectangle (16, -16);
%             \draw[dashed] (0,-4) node[left] {-4} -- (8.5,-4);
%             \draw[dashed,red] (16,-4.5) node[right] {-4.5} -- (8.5,-4.5);
%             \draw[dashed,red] (0,-6.5) node[left] {-6.5} -- (4,-6.5);
%             \draw[dashed,red] (16,-8.5) node[right] {-8.5} -- (8.5,-8.5);
%             \draw[dashed] (0,-11) node[left] {-11} -- (6.5,-11);
%             \draw[dashed] (16,-12) node[right] {-12} -- (12,-12);
%             \draw[dashed] (0,-13) node[left] {-13} -- (5,-13);
%             \draw[dashed,red] (4,0) node[above] {4} -- (4,-6.5);
%             \draw[dashed] (5,-16) node[below] {5} -- (5,-13);
%             \draw[dashed] (6.5,0) node[above] {6.5} -- (6.5,-11);
%             \draw[dashed] (8.5,-16) node[below] {8.5} -- (8.5,-13);
%             \draw[dashed] (9.5,0) node[above] {9.5} -- (9.5,-4);
%             \draw[dashed] (10.5,-16) node[below] {10.5} -- (10.5,-13);
%             \draw[dashed] (12,0) node[above] {12} -- (12,-12);
%         \end{scope}
%     \end{tikzpicture}
%     \caption*{示例: 旗子}
% \end{figure}
%
% \DescribeMacro{\mine} 在一个方格中绘制雷.
%
% \begin{figure}[!htp]
%     \centering
%     \begin{tikzpicture}
%         \begin{scope}[scale=10]
%             \mine
%         \end{scope}
%     \end{tikzpicture}
%     \caption*{示例: 雷}
% \end{figure}
%
% \DescribeMacro{\cellup} 绘制一个弹起的方格. 四周形成立体感的宽度为2.
%
% \DescribeMacro{\celldown} 绘制一个按下的方格. 左和上的边界宽度为1.
%
% \textbf{注意:} 弹起的方格和按下的方格的中心点不一致. 弹起的方格中心点为(8,-8), 按下的方格中心点为(8.5,-8.5).
%
% \begin{figure}[!htp]
%     \centering
%     \begin{tikzpicture}[scale=6]
%          \cellup
%          \draw[thick, cborder] (0,0) rectangle (16, -16);
%          \fill[red] (8, -8) circle (0.2);
%     \end{tikzpicture}
%     \quad
%     \begin{tikzpicture}[scale=6]
%          \celldown
%          \draw[thick, cborder] (0,0) rectangle (16, -16);
%          \fill[red] (8.5, -8.5) circle (0.2);
%     \end{tikzpicture}
%     \caption*{示例: 弹起的方格与按下的方格(包含中心点)}
% \end{figure}
%
% \DescribeMacro{\cellnum \marg{num}} 在当前方块中填入带配色的数字\raisebox{-0.75ex}{\tikz{\cellnum{0}}\tikz{\cellnum{1}}\tikz{\cellnum{2}}\tikz{\cellnum{3}}\tikz{\cellnum{4}}\tikz{\cellnum{5}}\tikz{\cellnum{6}}\tikz{\cellnum{7}}\tikz{\cellnum{8}}}.
%
% \textbf{注意:} 由于扫雷图形本身遵循的原则, 该命令不建议单独使用, 一般与|\celldown|指令一起使用.
%
% \subsection{边框元素}
% 边框分为三层, 中间层是color0, 宽度是6, 内外层用clight/cborder提供立体感, 宽度3. 提供立体感的方式和 |\cellup| 同理. 左边和上边的高亮外边缘用cborder封边.
%
% \DescribeMacro{\border \oarg{-tlbr}} 绘制以格为单位长度(宽度)的边框. t, l, b, r 分别代表上, 左, 下, 右边框, 如果相邻边框同时绘制, 将自动绘制对应的角.
%
% \begin{figure}[!htp]
%     \centering
%     \begin{tikzpicture}
%         \begin{scope}[scale=4]
%             \border
%             \draw[dashed] (-14,0) -- (30,0);
%             \draw[dashed] (-14,-16) -- (30,-16);
%             \draw[dashed] (0,14) -- (0,-30);
%             \draw[dashed] (16,14) -- (16,-30);
%             \node at (8, 6) {t};
%             \node at (22, -8) {r};
%             \node at (8, -22) {b};
%             \node at (-6, -8) {l};
%         \end{scope}
%     \end{tikzpicture}
%     \caption*{边框组装效果}
% \end{figure}
%
% \subsection{模块}
%
% \DescribeMacro{\cell \marg{r} \marg{c} \marg{info}} 在第[r]行第[c]列交叉处绘制包含内容[info]的方格, 行列从0开始编号. 如果内容是0-8的数字, 方格是按下状态并且数字会上色. 如果内容是f, 方格是弹起状态并且会插旗. 如果内容是其他单个字符, 方格是弹起状态并包含该字符. 方格弹起状态时的字符中心是(8,-8). 内容不能为空, 如果内容是-, 方格是按下状态并且无字符, 如果内容中含有多个字符, 将会自动缩小字符, 转化为 |\tiny| 模式. 如果内容是r, g, b, c, y, o, l, t, v, 将给该方格染成对应颜色.
%
% \DescribeMacro{\row \marg{r} \marg{seq:info}} 在第[r]行从左到右绘制多个方格. 内容的格式按照顺序输入字符, 每个字符作为 |\cell| 的第三个参数. 如果希望输入多个字符, 需要使用大括号, 例如|1{23}4|代表1, 23和4这这三个字符.
%
% \DescribeMacro{\col \marg{c} \marg{seq:info}}  在第[c]列从左到右绘制多个方格, 语法与 |row| 一致.
%
% \DescribeMacro{\board \oarg{-tlbr} \marg{r} \marg{c}} 绘制[r]行[c]列的边框, t, r, b, l 为与 \texttt{\textbackslash border} 对应\footnote{\texttt{\textbackslash board[-tlbr]\{1\}\{1\}}与\texttt{\textbackslash border[-tlbr]}等价.}的可叠加的边框开关, 默认全部开启, 即地图中有完整边框, 如果使用了\texttt{-}标志, 代表去除某个特定边框. 此选项可以根据自己需求进行定制\footnote{解析器只会解析最后一个\texttt{-}标志的位置, 即\texttt{\textbackslash border[-tlbr-tlb-lb-b]}会被解析为\texttt{\textbackslash border[-b]}. 解析器不会解析重复的参数, 即\texttt{\textbackslash border[ttblbtlb]}会被解析为\texttt{\textbackslash border[tlb]}.}.
%
% \begin{figure}[!htp]
%     \captionsetup[subfloat]{labelformat=empty}
%     \centering
%     \subfloat[\texttt{\textbackslash board}]{\begin{tikzpicture}[scale=2]\board{1}{1}\end{tikzpicture}}
%     \quad
%     \subfloat[\texttt{\textbackslash board[tlb]} 或 \\ \texttt{\textbackslash board[-r]} ]{\begin{tikzpicture}[scale=2]\border[-r]\end{tikzpicture}}
%     \quad
%     \subfloat[\texttt{\textbackslash board[tlr]} 或 \\ \texttt{\textbackslash board[-b]} ]{\begin{tikzpicture}[scale=2]\border[-b]\end{tikzpicture}}
%     \quad
%     \subfloat[\texttt{\textbackslash board[tl]}]{\begin{tikzpicture}[scale=2]\border[tl]\end{tikzpicture}}
% \end{figure}
%
% \DescribeMacro{\colorcell \marg{color} \\ \marg{seq: pos}} 将[pos]位置的方格染色为[color]颜色.
% \begin{multicols}{2}
% \iffalse
%<*internal>
% \fi
\begin{verbatim}
    \begin{tikzpicture}
        \board{3}{5}
        \row{0}{f{20}123}
        \row{1}{A-405}
        \row{2}{m-678}
        \colorcell{g}{0,0-1;0-1,3}
        \colorcell{r}{0,4;1-2,0-2}
    \end{tikzpicture}
\end{verbatim}
% \iffalse
%</internal>
% \fi
%     \columnbreak
%     \begin{figure}[H]
%         \centering
%         \begin{tikzpicture}
%             \board{3}{5}
%             \row{0}{f{20}123}
%             \row{1}{A-405}
%             \row{2}{m-678}
%             \colorcell{g}{
%                 0,0-1;0-1,3
%             }
%             \colorcell{r}{
%                 0,4;1-2,0-2
%             }
%         \end{tikzpicture}
%         \caption*{方格染色}
%     \end{figure}
% \end{multicols}
%
% \StopEventually{}
% \iffalse
%<*package>
\RequirePackage{tikz}
\RequirePackage{expl3}

\definecolor{color0}{RGB}{192,192,192} % color for blank cell
\definecolor{color1}{RGB}{0,0,255} % color for cell 1
\definecolor{color2}{RGB}{0,128,0} % color for cell 2
\definecolor{color3}{RGB}{255,0,0} % color for cell 3
\definecolor{color4}{RGB}{0,0,128} % color for cell 4
\definecolor{color5}{RGB}{122,43,26} % color for cell 5
\definecolor{color6}{RGB}{0,128,128} % color for cell 6
\definecolor{color7}{RGB}{0,0,0} % color for cell 7
\definecolor{color8}{RGB}{128,128,128} % color for cell 8
\definecolor{cborder}{RGB}{160,160,160} % color for border
\definecolor{clight}{RGB}{255,255,255} % color for whitespace
\definecolor{cshade}{RGB}{128,128,128} % color for shades
\definecolor{cfail}{RGB}{255,0,0} % color for failed cells

\tikzset{x=1pt, y=1pt} % Default unit

\ExplSyntaxOn
\cs_set:Nn \element_flag: {
    \fill[color7] (5, -13) rectangle (12, -12); % lower base
    \fill[color7] (6.5, -13) rectangle (10.5, -11); % upper base
    \fill[color7] (8.5, -13) rectangle (9.5, -4); % pole
    \fill[color3] (8.5, -4.5) -- (4, -6.5) -- (8.5, -8.5) -- cycle; % red flag
}

\cs_set:Nn \element_mine: {
    \int_step_inline:nnnn {0} {45} {135} {
        \fill[color7, xshift=8.5, yshift=-8.5, rotate=##1] (-6.5, 0.4) -- (0, 1.5) -- (6.5, 0.4) -- (6.5, -0.4) -- (0, -1.5) -- (-6.5, -0.4) -- cycle; % spikes
    }
    \shade[ball~color=color7] (8.5,-8.5) circle (4.5); % body
}

\cs_set:Nn \element_cellup: {
    \fill[clight] (0, 0) -- (16, 0) -- (14, -2) -- (2, -2) -- (2, -14) -- (0, -16) -- cycle; % highlight
    \fill[cshade] (16, -16) -- (0, -16) -- (2, -14) -- (14, -14) -- (14, -2) -- (16, 0) -- cycle; % shade
    \fill[color0] (2, -2) rectangle (14, -14); % background, modified to draw later
}

\cs_set:Nn \element_celldown: {
    \fill[color0] (1, -16) rectangle (16, -1); % background
    \fill[cshade] (0, 0) -- (16, 0) -- (16, -1) -- (1, -1) -- (1, -16) -- (0, -16) -- cycle; % border
}

\cs_set:Nn \element_cellfail: {
    \fill[cfail] (1, -16) rectangle (16, -1); % background
    \fill[cshade] (0, 0) -- (16, 0) -- (16, -1) -- (1, -1) -- (1, -16) -- (0, -16) -- cycle; % border
}

\cs_set:Npn \element_cellnum:n #1 {
    \tl_set:Nn \l_tmpa_tl {color}
    \tl_put_right:Nn \l_tmpa_tl {#1}
    \node[text~centered, text=\l_tmpa_tl] at (8.5, -8.5) {\textbf{#1}};
}

\cs_set:Npn \element_cellcolored:n #1 {
    \tl_case:NnT #1 {
        r { \tl_set:Nn \l_tmpa_tl {red} }
        g { \tl_set:Nn \l_tmpa_tl {green} }
        b { \tl_set:Nn \l_tmpa_tl {blue} }
        c { \tl_set:Nn \l_tmpa_tl {cyan} }
        y { \tl_set:Nn \l_tmpa_tl {yellow} }
        o { \tl_set:Nn \l_tmpa_tl {orange} }
        l { \tl_set:Nn \l_tmpa_tl {lime} }
        t { \tl_set:Nn \l_tmpa_tl {teal} }
        v { \tl_set:Nn \l_tmpa_tl {violet} }
    } {
        \begin{scope}
            \fill[color=\l_tmpa_tl, opacity=0.2] (0, 0) rectangle (16, -16);
        \end{scope}
    }
}

\cs_set:Npn \element_cell:nnn #1#2#3 {
    \sffamily
    \tl_set:Nn \l_tmpa_tl {012345678}
    \tl_set:Nn \l_tmpb_tl {rgbcyoltv}
    \begin{scope}[xshift=#3*16, yshift=-#2*16]
        \tl_if_single:nTF {#1} {
            % single token
            \tl_if_in:NnTF \l_tmpa_tl {#1} { \element_celldown: \element_cellnum:n {#1} } % number
            {
                \tl_if_in:NnTF \l_tmpb_tl {#1} { \element_cellup: \element_cellcolored:n {#1} } % color
                {
                    \tl_case:NnF #1 {
                        f { \element_cellup: \element_flag: } % flag
                        m { \element_celldown: \element_mine: } % mine
                        n { \element_cellfail: \element_mine: } % blasted mine
                        - { \element_cellup: } % empty
                    } { \element_cellup: \node[text~centered, text={color7}] at (8, -8) {\textbf{#1}}; } % normal label
                }
            }
        } { \element_cellup: \node[text~centered, text={color7}] at (8, -8) {\textbf{\tiny #1}}; } % tiny label
    \end{scope}
}

\cs_set:Npn \element_line:nn #1#2#3 {
    \tl_set:Nn \l_tmpa_tl {#2}
    \int_zero:N \l_tmpa_int
    \tl_map_inline:Nn \l_tmpa_tl {
        \tl_case:Nn #3 {
            r { \element_cell:nnn {##1}{#1}{\l_tmpa_int} }
            c { \element_cell:nnn {##1}{\l_tmpa_int}{#1} }
        }
        \int_incr:N \l_tmpa_int
    }
}

\cs_set:Npn \element_border:nnn #1#2#3 {
    \bool_set_false:N \l_tmpa_bool
    \bool_set_false:N \l__ms_flagt_bool
    \bool_set_false:N \l__ms_flagl_bool
    \bool_set_false:N \l__ms_flagb_bool
    \bool_set_false:N \l__ms_flagr_bool
    \tl_set:Nn \l_tmpa_tl {#1}

    % draw lines in advance to avoid gaps
    \int_step_inline:nnn {1} {#3-1} {
        \int_step_variable:nnNn {1} {#2} \l_tmpb_tl {
            \fill[cshade, xshift=##1*16, yshift=-\l_tmpb_tl*16] (0, 0) -- (-0.5, 0.5) -- (-0.5, 14.5) -- (0, 15) -- (-0.5, 15.5) -- (0, 16) -- (0.5, 15.5) -- (0, 15) -- (0.5, 14.5) -- (0.5, 0.5) -- cycle;
        }
    }
    \int_step_inline:nnn {1} {#2-1} {
        \int_step_variable:nnNn {1} {#3} \l_tmpb_tl {
            \fill[cshade, xshift=\l_tmpb_tl*16, yshift=-##1*16] (0, 0) -- (-0.5, -0.5) -- (-14.5, -0.5) -- (-15, 0) -- (-15.5, -0.5) -- (-16, 0) -- (-15.5, 0.5) -- (-15, 0) -- (-14.5, 0.5) -- (-0.5, 0.5) -- cycle;
        }
    }

    \tl_map_inline:Nn \l_tmpa_tl {
        \tl_case:Nn ##1 {
            - {
                \bool_set_true:N \l_tmpa_bool
                \bool_set_true:N \l__ms_flagt_bool
                \bool_set_true:N \l__ms_flagl_bool
                \bool_set_true:N \l__ms_flagb_bool
                \bool_set_true:N \l__ms_flagr_bool
            }
            t { \bool_set:Nn \l__ms_flagt_bool {!\l_tmpa_bool} }
            l { \bool_set:Nn \l__ms_flagl_bool {!\l_tmpa_bool} }
            b { \bool_set:Nn \l__ms_flagb_bool {!\l_tmpa_bool} }
            r { \bool_set:Nn \l__ms_flagr_bool {!\l_tmpa_bool} }
        }
    }

    \bool_if:NT \l__ms_flagt_bool {
        % side case: t
        \fill[clight] (0, 9) rectangle (#3*16, 12);
        \fill[color0] (0, 3) rectangle (#3*16, 9);
        \fill[cborder] (0, 0) rectangle (#3*16, 3);
        \fill[cborder] (0, 12) rectangle (#3*16, 12.1);
    }

    \bool_if:NT \l__ms_flagl_bool {
        % side case: l
        \fill[clight] (-9, 0) rectangle (-12, -#2*16);
        \fill[color0] (-3, 0) rectangle (-9, -#2*16);
        \fill[cborder] (0, 0) rectangle (-3, -#2*16);
        \fill[cborder] (-12, 0) rectangle (-12.1, -#2*16);
    }

    \bool_if:NT \l__ms_flagb_bool {
        % side case: b
        \begin{scope}[yshift=-#2*16]
            \fill[cborder] (0, -9) rectangle (#3*16, -12);
            \fill[color0] (0, -3) rectangle (#3*16, -9);
            \fill[clight] (0, 0) rectangle (#3*16, -3);
        \end{scope}
    }

    \bool_if:NT \l__ms_flagr_bool {
        % side case: r
        \begin{scope}[xshift=#3*16]
            \fill[cborder] (9, 0) rectangle (12, -#2*16);
            \fill[color0] (3, 0) rectangle (9, -#2*16);
            \fill[clight] (0, 0) rectangle (3, -#2*16);
        \end{scope}
    }

    \bool_lazy_and:nnT {\l__ms_flagt_bool} {\l__ms_flagl_bool} {
        % corner case: tl
        \fill[cborder] (-3, 3) -- (0, 3) -- (1.5, 1.5) -- (-1.5, -1.5) -- (-3, 0) -- cycle;
        \fill[color0] (-9, 9) -- (0, 9) -- (1.5, 6) -- (0, 3) -- (-3, 3) -- (-3, 0) -- (-6, -1.5) -- (-9, 0) -- cycle;
        \fill[clight] (-12, 12) -- (0, 12) -- (1.5, 10.5) -- (0, 9) -- (-9, 9) -- (-9, 0) -- (-10.5, -1.5) -- (-12, 0) -- cycle;
        \fill[cborder] (-12, 12) -- (0, 12) -- (0.05, 12.05) -- (0, 12.1) -- (-12.1, 12.1) -- (-12.1, 0) -- (-12.05, -0.05) -- (-12, 0) -- cycle;
    }

    \bool_lazy_and:nnT {\l__ms_flagb_bool} {\l__ms_flagl_bool} {
        % corner case: bl
        \begin{scope}[yshift=-#2*16]
            \fill[clight] (-3, -3) -- (0, -3) -- (1.5, -1.5) -- (0, 0) -- cycle;
            \fill[cborder] (-3, -3) -- (-3, 0) -- (-1.5, 1.5) -- (0, 0) -- cycle;
            \fill[color0] (-9, -9) -- (0, -9) -- (1.5, -6) -- (0, -3) -- (-3, -3) -- (-3, 0) -- (-6, 1.5) -- (-9, 0) -- cycle;
            \fill[clight] (-9, -9) -- (-12, -12) -- (-12, 0) -- (-10.5, 1.5) -- (-9, 0) -- cycle;
            \fill[cborder] (-9, -9) -- (-12, -12) -- (0, -12) -- (1.5, -10.5) -- (0, -9) -- cycle;
            \fill[cborder] (-12, -12) -- (-12, 0) -- (-12.05, 0.05) -- (-12.1, 0) -- (-12.1, -12) -- cycle;
        \end{scope}
    }

    \bool_lazy_and:nnT {\l__ms_flagb_bool} {\l__ms_flagr_bool} {
        % corner case: br
        \begin{scope}[xshift=#3*16, yshift=-#2*16]
            \fill[clight] (3, -3) -- (0, -3) -- (-1.5, -1.5) -- (1.5, 1.5) -- (3, 0) -- cycle;
            \fill[color0] (9, -9) -- (0, -9) -- (-1.5, -6) -- (0, -3) -- (3, -3) -- (3, 0) -- (6, 1.5) -- (9, 0) -- cycle;
            \fill[cborder] (12, -12) -- (0, -12) -- (-1.5, -10.5) -- (0, -9) -- (9, -9) -- (9, 0) -- (10.5, 1.5) -- (12, 0) -- cycle;
        \end{scope}
    }

    \bool_lazy_and:nnT {\l__ms_flagt_bool} {\l__ms_flagr_bool} {
        % corner case: tr
        \begin{scope}[xshift=#3*16]
            \fill[cborder] (3, 3) -- (0, 3) -- (-1.5, 1.5) -- (0, 0) -- cycle;
            \fill[clight] (3, 3) -- (3, 0) -- (1.5, -1.5) -- (0, 0) -- cycle;
            \fill[color0] (9, 9) -- (0, 9) -- (-1.5, 6) -- (0, 3) -- (3, 3) -- (3, 0) -- (6, -1.5) -- (9, 0) -- cycle;
            \fill[cborder] (9, 9) -- (12, 12) -- (12, 0) -- (10.5, -1.5) -- (9, 0) -- cycle;
            \fill[clight] (9, 9) -- (12, 12) -- (0, 12) -- (-1.5, 10.5) -- (0, 9) -- cycle;
            \fill[cborder] (12, 12) -- (0, 12) -- (-0.05, 12.05) -- (0, 12.1) -- (12, 12.1) -- cycle;
        \end{scope}
    }
}

\cs_set:Npn \element_colorcell:nn #1#2 {
    \tl_set:Nn \l_tmpa_tl {#2}
    \seq_set_split:NnV \l_tmpa_seq {;} \l_tmpa_tl
    \seq_map_inline:Nn \l_tmpa_seq {
        \tl_set:Nx \l_tmpb_tl {##1}
        \seq_set_split:NnV \l_tmpb_seq {,} \l_tmpb_tl
        \tl_set:Nx \l__ms_width_tl {\seq_item:Nn \l_tmpb_seq {1}}
        \tl_set:Nx \l__ms_height_tl {\seq_item:Nn \l_tmpb_seq {-1}}
        \seq_set_split:NnV \l__ms_width_seq {-} \l__ms_width_tl
        \seq_set_split:NnV \l__ms_height_seq {-} \l__ms_height_tl
        \tl_set:Nx \l__ms_widthx_tl {\seq_item:Nn \l__ms_width_seq {1}}
        \tl_set:Nx \l__ms_widthy_tl {\seq_item:Nn \l__ms_width_seq {-1}}
        \tl_set:Nx \l__ms_heightx_tl {\seq_item:Nn \l__ms_height_seq {1}}
        \tl_set:Nx \l__ms_heighty_tl {\seq_item:Nn \l__ms_height_seq {-1}}
        \int_step_variable:nnNn {\l__ms_widthx_tl} {\l__ms_widthy_tl} \l_tmpc_tl {
            \int_step_variable:nnNn {\l__ms_heightx_tl} {\l__ms_heighty_tl} \l_tmpd_tl {
                \begin{scope}[xshift=\l_tmpd_tl*16,yshift=-\l_tmpc_tl*16]
                    \element_cellcolored:n {#1}
                \end{scope}
            }
        }
    }
}

\newcommand{\flag}{\element_flag:}
\newcommand{\mine}{\element_mine:}
\newcommand{\cellup}{\element_cellup:}
\newcommand{\celldown}{\element_celldown:}
\newcommand{\cellnum}[1]{\element_cellnum:n {#1}}
\newcommand{\cell}[3]{\element_cell:nnn {#1}{#2}{#3}}
\newcommand{\row}[2]{\element_line:nn {#1}{#2}{r}}
\newcommand{\col}[2]{\element_line:nn {#1}{#2}{c}}
\newcommand{\border}[1][-]{\element_border:nnn {#1}{1}{1}}
\newcommand{\board}[3][-]{\element_border:nnn {#1}{#2}{#3}}
\newcommand{\colorcell}[2]{\element_colorcell:nn {#1}{#2}}

\ExplSyntaxOff

%</package>
% \fi
%
% \Finale
%
\endinput
