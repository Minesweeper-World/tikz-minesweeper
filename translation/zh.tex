%%
%% This is file `zh.tex',
%% Chinese version of the `tikz-minesweeper' package
%%
%% Copyright (C) 2021-2023 by Tian-Yi Pu, Fei-Yu Xiang
%% and Yao-Yu Zhu
%%
%% This file may be distributed and/or modified under the
%% conditions of the LaTeX Project Public License, either
%% version 1.3 of this license or (at your option) any later
%% version. The latest version of this license is in:
%%
%%    http://www.latex-project.org/lppl.txt
%%
%% and version 1.3 or later is part of all distributions of
%% LaTeX version 2005/12/01 or later.

\renewcommand{\figurename}{图}
\renewcommand{\tablename}{表}

\def\TransTitle{\pkg{tikz-minesweeper}宏包使用手册 (\fileversion)}
\def\TransAuthor{濮天羿~~~~向飞宇\thanks{源代码请参考此仓库: https://github.com/T0nyX1ang/tikz-minesweeper}~~~~朱耀宇}

\def\TransSectionUsage{绘制示例}
\def\TransSectionUsageContent{下面是一个\pkg{tikz-minesweeper}宏包的绘制示例,包含了该宏包中最重要的几个指令。}
\def\TransFigureUsageCaption{在5x5区域内绘制数字、雷、旗帜、空格和标记}

\def\TransSectionCommand{绘制指令介绍}

\def\TransSubsectionFundamentals{基础设定}
\def\TransSubsectionFundamentalsUnitContent{本宏包中设定单位长度为\textbf{1pt}。为了与标准扫雷游戏一致,一格的边长是\textbf{16pt}。下文如无特别说明,所有方格的原点都位于左上角,所有数据的单位均为\textbf{pt}。}
\def\TransSubsectionFundamentalsScaleContent{如果需要缩放一个盘面,推荐使用scope环境中的scale参数,即采用如下语法:}
\def\TransFigureZoomedCellCaption{放大一个格子}

\def\TransSubsectionColorScheme{配色方案}
\def\TransSubsectionColorSchemeContent{\pkg{tikz-minesweeper}宏包定义了用于绘制扫雷数字和盘面的配色。这些配色均给出了标签,这些标签均可以直接在文档中使用。}
\def\TransSubsectionColorSchemeTableContent{
    \begin{tabular}{ccccc}
        \toprule
        颜色标签           & R   & G   & B   & 场景                    \\
        \midrule
        \ColorTagZero{}   & 192 & 192 & 192 & 格子内部 / 外边框中间层   \\
        \ColorTagOne{}    & 0   & 0   & 255 & 数字1                   \\
        \ColorTagTwo{}    & 0   & 128 & 0   & 数字2                   \\
        \ColorTagThree{}  & 255 & 0   & 0   & 数字3                   \\
        \ColorTagFour{}   & 0   & 0   & 128 & 数字4                   \\
        \ColorTagFive{}   & 128 & 0   & 0   & 数字5                   \\
        \ColorTagSix{}    & 0   & 128 & 128 & 数字6                   \\
        \ColorTagSeven{}  & 0   & 0   & 0   & 数字7                   \\
        \ColorTagEight{}  & 128 & 128 & 128 & 数字8                   \\
        \ColorTagBorder{} & 160 & 160 & 160 & 边框阴影                 \\
        \ColorTagLight{}  & 255 & 255 & 255 & 与阴影对比的高亮颜色      \\
        \ColorTagShade{}  & 128 & 128 & 128 & 格子阴影 / 按下的格子边界  \\
        \ColorTagFail{}   & 255 & 0   & 0   & 导致失败的格子            \\
        \bottomrule
    \end{tabular}
}
\def\TransSubsectionColorSchemeTableCaption{颜色对照表}

\def\TransSubsectionCellElements{方格元素}
\def\TransCommandFlagContent{绘制旗子。首先用 \ColorTagSeven{} 绘制两层旗子的底座,然后用 \ColorTagSeven{} 绘制旗子的旗杆,最后用 \ColorTagThree{} 绘制旗帜本身。}
\def\TransCommandMineContent{绘制地雷。首先用 \ColorTagSeven{} 绘制地雷的梯形突起,然后用 \ColorTagSeven{} 绘制地雷的圆形本体与本体上的光影。}
\def\TransFigureFlagMineCaption{旗子和雷的细节图}
\def\TransCommandCellUpContent{绘制鼠标弹起时的方格。首先分别用 \ColorTagLight{} 和 \ColorTagShade{} 对称地绘制亮面和暗面,营造立体感,亮面和暗面的宽度均为2,然后用 \ColorTagZero{} 绘制底色。需要注意的是,该方格的中心点为 (8.5, -8.5),与 \tn{celldown} 指令对应方格的中心点不同。}
\def\TransCommandCellDownContent{绘制鼠标按下时的方格。首先用 \ColorTagShade{} 绘制方格边界,然后用 \ColorTagZero{} 绘制底色,左和上的边界宽度均为1。需要注意的是,c此时方格的中心点为 (8, -8),与 \tn{cellup} 指令对应方格的中心点不同。}
\def\TransFigureCellUpDownCaption{弹起的方格与按下的方格(包含中心点)}
\def\TransCommandCellNumContent{在当前方块中填入带配色的矢量化数字~\raisebox{-0.25ex}{\tikz{\cellnum{1}} \tikz{\cellnum{2}} \tikz{\cellnum{3}} \tikz{\cellnum{4}} \tikz{\cellnum{5}} \tikz{\cellnum{6}} \tikz{\cellnum{7}} \tikz{\cellnum{8}}}。如果 \meta{num} 不为 1 至 8,则该指令不起作用。需要注意的是,由于扫雷图形本身遵循的原则,该命令不建议单独使用,一般与 \tn{celldown} 指令一起使用。}

\def\TransSubsectionFrameElements{边框元素}
\def\TransCommandBorderContent{绘制以格为单位长度(宽度)的边框。\oarg{t}、\oarg{l}、\oarg{b}、\oarg{r} 分别代表上、左、下、右边框,如果相邻边框同时绘制,将自动绘制对应的角。边框分为三层,中间层使用 \ColorTagZero{} 绘制,宽度为 6,内外层用 \ColorTagLight{} 与 \ColorTagBorder{} 提供立体感,宽度为 3。提供立体感的方式和 \tn{cellup} 同理。左边和上边的高亮外边缘用 \ColorTagBorder{} 封边。默认参数为 \oarg{-},即绘制全部边框。}
\def\TransFigureBorderCaption{边框组装效果}

\def\TransSubsectionModules{综合模块}
\def\TransSubsectionModulesContent{这一部分的模块是 \pkg{tikz-minesweeper} 提供给用户使用的模块,如果没有特定的设计需求,推荐仅使用这一部分提供的模块。}
\def\TransCommandCellContent{在第 \meta{r} 行第 \meta{c} 列交叉处绘制包含内容 \meta{info} 的方格,行列均从 0 开始编号。\meta{info} 的检查顺序如下:
    \begin{itemize}
        \item 如果 \meta{info} 是 \marg{0}-\marg{8} 的数字,方格是按下状态并且数字会上色;
        \item 如果 \meta{info} 是 \marg{r}(red,红色)、\marg{g}(绿色)、\marg{b}(蓝色)、\marg{c}(cyan,湖蓝色)、\marg{y}(yellow,黄色)、\marg{o}(orange,橘色)、\marg{v}(violet,紫色)中的一个,方格是抬起状态,且会染成对应颜色,方格颜色的透明度为\textbf{0.2};
        \item 如果 \meta{info} 是 \marg{f},方格是弹起状态并且会插旗;
        \item 如果 \meta{info} 是 \marg{m},方格是按下状态并且会显示地雷;
        \item 如果 \meta{info} 是 \marg{s},方格是按下状态并且会显示半透明的地雷,即将地雷的颜色调整为 \ColorTagShade{};
        \item 如果 \meta{info} 是 \marg{n},方格是按下状态并且会显示踩到地雷,用 \ColorTagFail{} 绘制底色;
        \item 如果 \meta{info} 是 \marg{e},方格是按下状态并且会显示标记错误的地雷,即在普通地雷的基础上,进一步用 \ColorTagFail{} 绘制一个叉;
        \item 如果 \meta{info} 是 \marg{-},方格是弹起状态并且无字符;
        \item 如果 \meta{info} 是其他\textbf{单个}字符,方格是弹起状态并包含该字符,字符中心为 (8, -8);
        \item 如果 \meta{info} 包含\textbf{多个}字符,将会自动缩小字符,转化为 \tn{tiny} 模式。
    \end{itemize}
}
