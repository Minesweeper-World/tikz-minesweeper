%%
%% This is file `zh.tex',
%% Chinese version of the `tikz-minesweeper' package
%%
%% Copyright (C) 2021-2023 by Tian-Yi Pu, Fei-Yu Xiang
%% and Yao-Yu Zhu
%%
%% This file may be distributed and/or modified under the
%% conditions of the LaTeX Project Public License, either
%% version 1.3 of this license or (at your option) any later
%% version. The latest version of this license is in:
%%
%%    http://www.latex-project.org/lppl.txt
%%
%% and version 1.3 or later is part of all distributions of
%% LaTeX version 2005/12/01 or later.

\renewcommand{\figurename}{图}
\renewcommand{\tablename}{表}
\renewcommand{\indexname}{索引}
\def\TransIndexPrologueContent{\emph{意大利体的数字表示描述对应索引项的页码, 带下划线的数字表示定义对应索引项的代码行号, 罗马字体的数字表示使用对应索引项的代码行号.}}

\def\TransTitle{\pkg{tikz-minesweeper}宏包使用手册 (\fileversion)}
\def\TransAuthor{濮天羿~~~~向飞宇\thanks{源代码请参考此仓库: https://github.com/T0nyX1ang/tikz-minesweeper}~~~~朱耀宇}

\def\TransSectionUsage{绘制示例}
\def\TransSectionUsageContent{下面是一个\pkg{tikz-minesweeper}宏包的绘制示例, 包含了该宏包中最重要的几个指令.}
\def\TransFigureUsageCaption{在5x5区域内绘制数字、雷、旗帜、空格和标记}

\def\TransSectionCommand{绘制指令介绍}

\def\TransSubsectionFundamentals{基础设定}
\def\TransSubsectionFundamentalsUnitContent{本宏包中设定单位长度为\textbf{1pt}. 为了与标准扫雷游戏一致, 一格的边长是\textbf{16pt}. 下文如无特别说明, 所有方格的原点都位于左上角.}
\def\TransSubsectionFundamentalsScaleContent{如果需要缩放一个盘面, 推荐使用scope环境中的scale参数, 即采用如下语法:}
\def\TransFigureZoomedCellCaption{两倍放大一个格子}

\def\TransSubsectionColorScheme{配色方案}
\def\TransSubsectionColorSchemeContent{\pkg{tikz-minesweeper}宏包定义了用于绘制扫雷数字和盘面的配色. 这些配色均给出了标签, 这些标签均可以直接在文档中使用.}
\def\TransSubsectionColorSchemeTableContent{
    \begin{tabular}{ccccc}
        \toprule
        颜色标签           & R   & G   & B   & 场景                    \\
        \midrule
        \ColorTagZero{}   & 192 & 192 & 192 & 格子内部 / 外边框中间层   \\
        \ColorTagOne{}    & 0   & 0   & 255 & 数字1                   \\
        \ColorTagTwo{}    & 0   & 128 & 0   & 数字2                   \\
        \ColorTagThree{}  & 255 & 0   & 0   & 数字3                   \\
        \ColorTagFour{}   & 0   & 0   & 128 & 数字4                   \\
        \ColorTagFive{}   & 128 & 0   & 0   & 数字5                   \\
        \ColorTagSix{}    & 0   & 128 & 128 & 数字6                   \\
        \ColorTagSeven{}  & 0   & 0   & 0   & 数字7                   \\
        \ColorTagEight{}  & 128 & 128 & 128 & 数字8                   \\
        \ColorTagBorder{} & 160 & 160 & 160 & 边框阴影                 \\
        \ColorTagLight{}  & 255 & 255 & 255 & 与阴影对比的高亮颜色      \\
        \ColorTagShade{}  & 128 & 128 & 128 & 格子阴影 / 按下的格子边界  \\
        \ColorTagFail{}   & 255 & 0   & 0   & 导致失败的格子            \\
        \bottomrule
    \end{tabular}
}
\def\TransSubsectionColorSchemeTableCaption{颜色对照表}

\def\TransSubsectionCellElements{方格元素}
\def\TransCommandCellContent{在第 \meta{r} 行第 \meta{c} 列交叉处绘制包含内容 \meta{info} 的方格, 行列均从 0 开始编号. \meta{info} 的检查顺序如下:
    \begin{optdesc}
        \item[0] 如果 \meta{info} 是 \meta{0}, 仅绘制一个按下状态的方格. 按下状态的方格首先用 \ColorTagShade{} 绘制方格边界, 然后用 \ColorTagZero{} 绘制底色, 左和上的边界宽度均为 1pt, 此时方格的中心点为 (8, -8);
        \item[1{\textbar}2{\textbar}3{\textbar}4{\textbar}5{\textbar}6{\textbar}7{\textbar}8] 如果 \meta{info} 是 \meta{1}-\meta{8}, 方格是按下状态, 并且填入带配色的矢量化数字;
        \item[r{\textbar}g{\textbar}b{\textbar}c{\textbar}y{\textbar}o{\textbar}v] 如果 \meta{info} 是 \meta{r} (red, 红色)、\meta{g} (绿色)、\meta{b} (蓝色)、\meta{c} (cyan, 湖蓝色)、\meta{y} (yellow, 黄色)、\meta{o} (orange, 橘色)、\meta{v} (violet, 紫色)中的一个, 方格是抬起状态, 且会染成对应颜色, 方格颜色的透明度为 0.2;
        \item[f] 如果 \meta{info} 是 \meta{f}, 方格是弹起状态并且会插旗. 旗帜首先用 \ColorTagSeven{} 绘制两层旗子的底座, 然后用 \ColorTagSeven{} 绘制旗子的旗杆, 最后用 \ColorTagThree{} 绘制旗帜本身;
        \item[m] 如果 \meta{info} 是 \meta{m}, 方格是按下状态并且会显示地雷. 地雷首先用 \ColorTagSeven{} 绘制地雷的梯形突起, 然后用 \ColorTagSeven{} 绘制地雷的圆形本体与本体上的光影;
        \item[s] 如果 \meta{info} 是 \meta{s}, 方格是按下状态并且会显示半透明的地雷, 即将地雷的颜色调整为 \ColorTagShade{};
        \item[n] 如果 \meta{info} 是 \meta{n}, 方格是按下状态并且会显示踩到地雷, 用 \ColorTagFail{} 绘制底色;
        \item[e] 如果 \meta{info} 是 \meta{e}, 方格是按下状态并且会显示标记错误的地雷, 即在普通地雷的基础上, 进一步用 \ColorTagFail{} 绘制一个叉;
        \item[-] 如果 \meta{info} 是 \meta{-}, 方格是弹起状态并且无字符. 弹起状态的方格首先分别用 \ColorTagLight{} 和 \ColorTagShade{} 对称地绘制亮面和暗面, 营造立体感, 亮面和暗面的宽度均为 2pt, 然后用 \ColorTagZero{} 绘制底色. 需要注意的是, 该方格的中心点为 (8.5, -8.5);
        \item[(?)] 如果 \meta{info} 是其他\textbf{单个}字符, 方格是弹起状态并包含该字符, 字符中心为 (8, -8);
        \item[(*)] 如果 \meta{info} 包含\textbf{多个}字符, 将会自动缩小字符, 转化为 \tn[no-index]{tiny} 模式.
    \end{optdesc}
    相关元素绘制细节见图\ref{fig-flag-mine}和图\ref{fig-cellup-celldown}.
}
\def\TransFigureFlagMineNumCaption{旗子、雷和数字的细节图}
\def\TransFigureCellUpDownCaption{弹起的方格与按下的方格 (包含中心点)}

\def\TransSubsectionBorderElements{边框元素}
\def\TransCommandBorderContent{绘制 \meta{r} 行, \meta{c} 列的边框. 边框分为三层, 中间层使用 \ColorTagZero{} 绘制, 宽度为 6pt, 内外层用 \ColorTagLight{} 与 \ColorTagBorder{} 提供立体感, 宽度为 3pt. 提供立体感的方式和弹起方格同理. 左边和上边的高亮外边缘用 \ColorTagBorder{} 封边. 边框选项 \meta{border\_option} 包含:
    \begin{optdesc}
        \item[-] 这是取反选项, 默认为绘制完整的边框. 如果使用了 \meta{-} 标志, 代表去除某些特定方位的边框. 此选项可以根据自己需求进行定制. 解析器只会解析最后一个 \meta{-} 标志的位置, 即 \tn[no-index]{border[-br-tl-lb-b]} 会被解析为 \tn[no-index]{border[-b]}.
        \item[t{\textbar}l{\textbar}b{\textbar}r] 这是方位选项. \meta{t}、\meta{l}、\meta{b}、\meta{r} 分别代表上、左、下、右边框. 此外, 解析器不会解析重复的参数, 即 \tn[no-index]{border[ttblbtlb]} 会被解析为 \tn[no-index]{border[tlb]}.
        \item[tl{\textbar}tr{\textbar}bl{\textbar}br] 如果相邻的方位选项同时出现, 将同时绘制对应的边框角落.
    \end{optdesc}
    边框与边框选项的细节见图\ref{fig-border-full}和图\ref{fig-border-options}.
}
\def\TransFigureBorderCaption{边框组装效果}
\def\TransFigureBorderOptionCaption{边框选项示例}

\def\TransSubsectionModules{综合模块}
\def\TransSubsectionModulesContent{这一部分的模块是 \pkg{tikz-minesweeper} 提供给用户使用的模块, 如果没有特定的设计需求, 推荐仅使用这一部分提供的模块.}
\def\TransCommandRowContent{在第 \meta{r} 行从左到右绘制多个方格. \meta{seq:info} 为一个含有 \meta{info} 的序列, 序列中的每个元素作为 \tn{cell} 的第三个参数. 如果希望输入多个字符, 需要使用大括号, 例如 \meta{1\{23\}4} 的序列长度为3, 该序列中的元素分别为 \meta{1}, \meta{23} 和 \meta{4}.}
\def\TransCommandBoardContent{绘制 \meta{r} 行 \meta{c} 列的盘面, 盘面选项中的 \meta{t}, \meta{r}, \meta{b}, \meta{l} 与 \tn{border} 的边框选项一致. 选项 \meta{x} 表示是否显示横纵坐标, 默认为不显示横纵坐标. 在绘制边框的过程中, 该指令会根据边框类型, 自动以 \ColorTagBorder{} 为颜色, 绘制一个 0.2pt 的外边框. 同时, 为了避免格子的同色邻接区域导致的渲染问题, 该指令会提前绘制同色邻接区域.}
\def\TransFigureBoardCoordinateCaption{横纵坐标显示示例}
\def\TransCommandColorCellContent{将 \meta{seq: pos} 位置的方格染色为 \meta{color} 颜色. \meta{color} 仅能传入 \tn{cell} 中接受的颜色类型, 但是与 \tn{cell} 仅支持单一颜色不同的是, \meta{color} 支持填入至多 4 种不同颜色, 颜色将从方格顶部逆时针填充. \meta{seq: pos} 定义了一个位置序列, 格式如下:
    \begin{itemize}
        \item 序列中的每个元素以分号(;)分隔.
        \item 对于每个元素, 以逗号(,)分隔行区间和列区间.
        \item 对于每个区间, 如果它是\textbf{单字符}, 则表示区间的起始位置和终止位置相同, 均为该字符; 否则需要以减号(-)分隔起始位置和终止位置.
    \end{itemize}
}
\def\TransFigureColorCellCaption{方格染色示例}

\def\TransSectionImplementation{代码实现}
\def\TransMacroDependencyContent{定义 \pkg{tikz-minesweeper} 宏包的依赖, 分别为 \LaTeXiii{} 和 \pkg{tikz} 宏包.}
\def\TransMacroColorSchemeContent{定义 \pkg{tikz-minesweeper} 宏包的配色方案. 在实际使用该宏包时, 可以使用类似方法定义自己的配色方案.}
\def\TransMacroUnitContent{定义 \pkg{tikz-minesweeper} 宏包的基本单位, 即 1pt.}
\def\TransMacroEnableLaTeXiiiEnvContent{打开\LaTeXiii{}编程环境.}
\def\TransMacroVariableDefinitionContent{定义 \pkg{tikz-minesweeper} 宏包中的变量.}
\def\TransMacroValidateInRangeContent{判断 \meta{\#1} 是否介于下界 \meta{\#2} 和上界 \meta{\#3} 之间, 包含边界. 如果下界 \meta{\#2} 为空值, 则仅判断 \meta{\#1} 是否小于上界 \meta{\#3}; 如果上界 \meta{\#3} 为空值, 则仅判断 \meta{\#1} 是否大于上界 \meta{\#2}.}
\def\TransMacroFlagContent{绘制旗子.}
\def\TransMacroMineContent{绘制地雷.}
\def\TransMacroMineShadedContent{绘制半透明的地雷. 需要注意的是, 半透明的雷的颜色没有与其它颜色解耦, 如果定义了不同的配色方案, 请谨慎绘制这种地雷.}
\def\TransMacroCellUpContent{绘制鼠标弹起时的方格.}
\def\TransMacroCellDownContent{绘制鼠标按下后的方格.}
\def\TransMacroCellFailContent{绘制由于踩雷而失败的方格.}
\def\TransMacroMisflagFailContent{绘制由于标记错误而失败的方格.}
\def\TransMacroCellNumContent{用 \ColorTagOne{} 至 \ColorTagEight{} 绘制给定的矢量化数字 1 至 8, 接受一个参数 \meta{\#1}. 参数 \meta{\#1} 仅接受 \meta{12345678} 中的字符, 传入其它值将会被忽略.}
\def\TransMacroCellColoredContent{绘制鼠标按下时的染色方格, 接受一个参数 \meta{\#1}. 参数 \meta{\#1} 仅接受 \meta{rgbcyov} 中的字符, 传入其它值将会被忽略. 参数 \meta{\#1} 的最长长度为 4, 如果传入的参数长度小于 4, 则会在参数的左/右侧补充与其相同的字符, 如果传入的参数长度大于 4, 则会直接被截断为 4 个字符.}
\def\TransMacroCellContent{绘制方格, 接受三个参数 \meta{\#1}, \meta{\#2}, \meta{\#3}. 在坐标(\meta{\#3}, \meta{\#2})处绘制一个方格, 方格中的内容由 \meta{\#1} 决定, \meta{\#1}可以传入任意参数. 该函数依次检查 \meta{\#1} 是否为数字参数 \meta{012345678}, 颜色参数 \meta{rgbcyov}, 雷与旗帜参数 \meta{fmsne-}, 除上述字符外的单字符参数以及多字符参数.}
\def\TransMacroRowContent{绘制一行格子, 接受两个参数 \meta{\#1}, \meta{\#2}. \meta{\#1} 为行号, \meta{\#2} 为格子内容.}
\def\TransMacroRowColMarkerContent{绘制行列坐标数字.}
\def\TransMacroGapFillerContent{由于渲染问题, 绘制盘面时, 同色区域之间可能会出现一个空白的缝隙, 该函数用于提前填充这些缝隙.}
\def\TransMacroBorderContent{绘制边框, 接受三个参数 \meta{\#1}, \meta{\#2}, \meta{\#3}. \meta{\#1} 为边框种类参数, \meta{\#2} 为行数, \meta{\#3} 为列数. 参数 \meta{\#1} 仅接受 \meta{-tlbr} 中的值.}
\def\TransMacroBoardContent{绘制盘面, 接受三个参数 \meta{\#1}, \meta{\#2}, \meta{\#3}. \meta{\#1} 为边框种类参数, \meta{\#2} 为行数, \meta{\#3} 为列数. 参数 \meta{\#1} 仅接受 \meta{-tlbrx} 中的值.}
\def\TransMacroColorCellContent{给多个区域染色, 接受两个参数 \meta{\#1}, \meta{\#2}. \meta{\#1} 仅接受 \cs[no-index]{msweeper_cellcolored:n} 中的颜色, \meta{\#2} 传入待绘制的多个区域.}
\def\TransMacroUserAPIContent{提供不显式调用 \LaTeXiii{} 的用户接口.}
\def\TransMacroDisableLaTeXiiiEnvContent{关闭 \LaTeXiii{} 编程环境.}
