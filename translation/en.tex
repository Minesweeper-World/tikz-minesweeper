%%
%% This is file `zh.tex',
%% English version of the `tikz-minesweeper' package's documents
%%
%% Copyright (C) 2021-2023 by Tian-Yi Pu, Fei-Yu Xiang
%% and Yao-Yu Zhu
%%
%% This file may be distributed and/or modified under the
%% conditions of the LaTeX Project Public License, either
%% version 1.3 of this license or (at your option) any later
%% version. The latest version of this license is in:
%%
%%    http://www.latex-project.org/lppl.txt
%%
%% and version 1.3 or later is part of all distributions of
%% LaTeX version 2005/12/01 or later.

\renewcommand{\figurename}{Figure}
\renewcommand{\tablename}{Table}

\def\TransTitle{\pkg{tikz-minesweeper} Manual (\fileversion)}
\def\TransAuthor{Tian-Yi Pu~~~~Fei-Yu Xiang \thanks{Source code: https://github.com/T0nyX1ang/tikz-minesweeper}~~~~Yao-Yu Zhu}

\def\TransSectionUsage{Usage}
\def\TransSectionUsageContent{An example including the most important commands within the \pkg{tikz-minesweeper} package is as follows.}
\def\TransFigureUsageCaption{Rendering numbers, mines, flags, cells and marks on a 5x5 board}

\def\TransSectionCommand{Reference of commands}

\def\TransSubsectionFundamentals{Fundamental rules}
\def\TransSubsectionFundamentalsUnitContent{This package sets the unit length to \textbf{1pt}. The edge length for cells is \textbf{16pt} to match the standard minesweeper game. Henceforth, without explicit note, the origin of a cell is its top-left corner and all lengths are in \textbf{pt}.}
\def\TransSubsectionFundamentalsScaleContent{To scale a board, it's recommended to use the scale parameter from scope environment, that is: }
\def\TransFigureZoomedCellCaption{A zoomed cell}

\def\TransSubsectionColorScheme{Color scheme}
\def\TransSubsectionColorSchemeContent{The \pkg{tikz-minesweeper} package defines the colors for numbers and board margins in minesweeper. The colors have labels that can be directly executed in the documentation.}
\def\TransSubsectionColorSchemeTableContent{
    \begin{tabular}{ccccc}
        \toprule
        Color label       & R   & G   & B   & Use case                            \\
        \midrule
        \ColorTagZero{}   & 192 & 192 & 192 & cell interior/middle layer of frame \\
        \ColorTagOne{}    & 0   & 0   & 255 & number 1                            \\
        \ColorTagTwo{}    & 0   & 128 & 0   & number 2                            \\
        \ColorTagThree{}  & 255 & 0   & 0   & number 3                            \\
        \ColorTagFour{}   & 0   & 0   & 128 & number 4                            \\
        \ColorTagFive{}   & 128 & 0   & 0   & number 5                            \\
        \ColorTagSix{}    & 0   & 128 & 128 & number 6                            \\
        \ColorTagSeven{}  & 0   & 0   & 0   & number 7                            \\
        \ColorTagEight{}  & 128 & 128 & 128 & number 8                            \\
        \ColorTagBorder{} & 160 & 160 & 160 & frame shade                         \\
        \ColorTagLight{}  & 255 & 255 & 255 & highlight in contrast with shades   \\
        \ColorTagShade{}  & 128 & 128 & 128 & cell shade/border of opened cell    \\
        \ColorTagFail{}   & 255 & 0   & 0   & blasted cell                        \\
        \bottomrule
    \end{tabular}
}
\def\TransSubsectionColorSchemeTableCaption{Look-up table of colors}

\def\TransSubsectionCellElements{Cell Elements}
\def\TransCommandFlagContent{Draw the flag. Draw the 2-layer base of the flag with \ColorTagSeven{} first, then draw the flag pole with \ColorTagSeven{}. In the end, draw the flag with \ColorTagThree{}.}
\def\TransCommandMineContent{Draw the mine. Draw the trapezoidal spikes with \ColorTagSeven{} first, then draw the circular body and the highlight with \ColorTagSeven{}.}
\def\TransFigureFlagMineCaption{Sketch of the flag and the mine}
\def\TransCommandCellUpContent{Draw the covered cell. Draw the light side and dark side, the widths of which are both 2, with \ColorTagLight{} and \ColorTagShade{} respectively to create a 3D effect. Then fill the background with \ColorTagZero{}. Note that the center of this cell is (8.5, -8.5), which is different from that of \tn{celldown}.}
\def\TransCommandCellDownContent{Draw the revealed cell. Draw the edges with \ColorTagShade{}, then fill the background with \ColorTagZero{}. The widths of the left and top edges are both 1. Note that the center of this cell is (8, -8), which is different from that of \tn{cellup}.}
\def\TransFigureCellUpDownCaption{Covered cell and revealed cell (the centers are marked)}
\def\TransCommandCellNumContent{Fill the cell with colored vectorized number~\raisebox{-0.25ex}{\tikz{\cellnum{1}} \tikz{\cellnum{2}} \tikz{\cellnum{3}} \tikz{\cellnum{4}} \tikz{\cellnum{5}} \tikz{\cellnum{6}} \tikz{\cellnum{7}} \tikz{\cellnum{8}}}. If \meta{num} is not from 1 to 8, then the command will not take effect. Note that this command is usually used with \tn{celldown}. It's not recommended using this command alone due to the general rule of minesweeper graphics. }

\def\TransSubsectionFrameElements{Frame elements}
\def\TransCommandBorderContent{Draw the cell-sized frame. \oarg{t}, \oarg{l}, \oarg{b}, \oarg{r} represent the top, left, bottom, right frame respectively. If two adjacent sides are drawn at the same time, the corner in between will also be drawn. The frame consists of 3 layers. The middle layer is in \ColorTagZero{} and 6 units wide. The inner and outer layer are 3 units wide, using \ColorTagLight{} and \ColorTagBorder{} to create a 3D effect similarly as \tn{cellup}. The outer edge of the left and top highlights are enclosed with \ColorTagBorder{}. The default argument is \oarg{-}, which means drawing the full frame.}
\def\TransFigureBorderCaption{The full frame in effect}

\def\TransSubsectionModules{Modules}
\def\TransSubsectionModulesContent{This section contains the modules provided by \pkg{tikz-minesweeper}. In general, it's recommended to only use these commands unless there is specific customization needed.}
\def\TransCommandCellContent{Draw a cell containing \meta{info} at the intersection of the \meta{r}-th row and the \meta{c}-th column, starting from 0. \meta{info} is checked in the following order:
    \begin{itemize}
        \item If \meta{info} is a number from \marg{0} to \marg{8}, then the cell is pressed down and the number is colored;
        \item If \meta{info} is \marg{r} (red), \marg{g} (green), \marg{b} (blue), \marg{c} (cyan), \marg{y} (yellow), \marg{o} (orange) or \marg{v} (violet), then the cell is covered and accordingly colored with a transparency of \textbf{0.2};
        \item If \meta{info} is \marg{f}, then the cell is covered and flagged;
        \item If \meta{info} is \marg{m}, then the cell is pressed and shows a mine;
        \item If \meta{info} is \marg{s}, then the cell is pressed and shows a semitransparent mine, which is basically a mine in \ColorTagShade{};
        \item If \meta{info} is \marg{n}, then the cell is pressed and shows a blasted mine, with \ColorTagFail{} for background;
        \item If \meta{info} is \marg{e}, then the cell is pressed and shows a mistakenly flagged mine, which is a cross in \ColorTagFail{} on top of a mine;
        \item If \meta{info} is \marg{-}, then the cell is covered and has no content;
        \item If \meta{info} is any other \textbf{single} character, then the cell is covered and shows the character centered at (8, -8);
        \item If \meta{info} contains \textbf{multiple} characters, the string will be shown in \tn{tiny} font.
    \end{itemize}
}
