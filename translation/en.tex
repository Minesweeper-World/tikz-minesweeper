%%
%% This is file `zh.tex',
%% English version of the `tikz-minesweeper' package's documents
%%
%% Copyright (C) 2021-2023 by Tian-Yi Pu, Fei-Yu Xiang
%% and Yao-Yu Zhu
%%
%% This file may be distributed and/or modified under the
%% conditions of the LaTeX Project Public License, either
%% version 1.3 of this license or (at your option) any later
%% version. The latest version of this license is in:
%%
%%    http://www.latex-project.org/lppl.txt
%%
%% and version 1.3 or later is part of all distributions of
%% LaTeX version 2005/12/01 or later.

\renewcommand{\figurename}{Figure}
\renewcommand{\tablename}{Table}
\renewcommand{\indexname}{Index}
\def\TransIndexPrologueContent{\emph{The italic numbers denote the pages where the corresponding entry is described, numbers underlined point to the definition, all others indicate the places where it is used.}}

\def\TransTitle{\pkg{tikz-minesweeper} Manual (\fileversion)}
\def\TransAuthor{Tian-Yi Pu~~~~Fei-Yu Xiang \thanks{Source code: https://github.com/T0nyX1ang/tikz-minesweeper}~~~~Yao-Yu Zhu}

\def\TransSectionUsage{Usage}
\def\TransSectionUsageContent{An example including the most important commands within the \pkg{tikz-minesweeper} package is as follows.}
\def\TransFigureUsageCaption{Rendering numbers, mines, flags, cells and marks on a 5x5 board}

\def\TransSectionCommand{Reference of commands}

\def\TransSubsectionFundamentals{Fundamental rules}
\def\TransSubsectionFundamentalsUnitContent{This package sets the unit length to \textbf{1pt}. The edge length for cells is \textbf{16pt} to match the standard minesweeper game. Henceforth, without explicit note, the origin of a cell is its top-left corner.}
\def\TransSubsectionFundamentalsScaleContent{To scale a board, it's recommended to use the scale parameter from scope environment, that is: }
\def\TransFigureZoomedCellCaption{A zoomed cell by two times}

\def\TransSubsectionColorScheme{Color scheme}
\def\TransSubsectionColorSchemeContent{The \pkg{tikz-minesweeper} package defines the colors for numbers and board margins in minesweeper. The colors have labels that can be directly executed in the documentation.}
\def\TransSubsectionColorSchemeTableContent{
    \begin{tabular}{ccccc}
        \toprule
        Color label       & R   & G   & B   & Use case                            \\
        \midrule
        \ColorTagZero{}   & 192 & 192 & 192 & cell interior/middle layer of frame \\
        \ColorTagOne{}    & 0   & 0   & 255 & number 1                            \\
        \ColorTagTwo{}    & 0   & 128 & 0   & number 2                            \\
        \ColorTagThree{}  & 255 & 0   & 0   & number 3                            \\
        \ColorTagFour{}   & 0   & 0   & 128 & number 4                            \\
        \ColorTagFive{}   & 128 & 0   & 0   & number 5                            \\
        \ColorTagSix{}    & 0   & 128 & 128 & number 6                            \\
        \ColorTagSeven{}  & 0   & 0   & 0   & number 7                            \\
        \ColorTagEight{}  & 128 & 128 & 128 & number 8                            \\
        \ColorTagBorder{} & 160 & 160 & 160 & frame shade                         \\
        \ColorTagLight{}  & 255 & 255 & 255 & highlight in contrast with shades   \\
        \ColorTagShade{}  & 128 & 128 & 128 & cell shade/border of opened cell    \\
        \ColorTagFail{}   & 255 & 0   & 0   & blasted cell                        \\
        \bottomrule
    \end{tabular}
}
\def\TransSubsectionColorSchemeTableCaption{Look-up table of colors}

\def\TransSubsectionCellElements{Cell Elements}
\def\TransCommandCellContent{Draw the cell containing \meta{info} at the intersection of the \meta{r}-th row and the \meta{c}-th column, starting from 0. \meta{info} is checked in the following order:
    \begin{optdesc}
        \item[0] If \meta{info} is \meta{0}, then the cell is only pressed down. A pressed cell is drawn the edges with \ColorTagShade{}, then the background is filled with \ColorTagZero{}. The widths of the left and top edges are both 1pt. Note that the center of this cell is (8, -8);
        \item[1{\textbar}2{\textbar}3{\textbar}4{\textbar}5{\textbar}6{\textbar}7{\textbar}8] If \meta{info} is from \meta{0} to \meta{8}, then the cell is pressed and the vectorized numbers are drawn;
        \item[r{\textbar}g{\textbar}b{\textbar}c{\textbar}y{\textbar}o{\textbar}v] If \meta{info} is \meta{r} (red), \meta{g} (green), \meta{b} (blue), \meta{c} (cyan), \meta{y} (yellow), \meta{o} (orange) or \meta{v} (violet), then the cell is covered and accordingly colored with a transparency of 0.2;
        \item[f] If \meta{info} is \meta{f}, then the cell is covered and flagged. The 2-layer base of the flag is colored with \ColorTagSeven{} first, then the flag pole is colored with \ColorTagSeven{}. In the end, the flag is colored with \ColorTagThree{};
        \item[m] If \meta{info} is \meta{m}, then the cell is pressed and shows a mine. Trapezoidal spikes are colored with \ColorTagSeven{} first, then the circular body is shaded with \ColorTagSeven{};
        \item[s] If \meta{info} is \meta{s}, then the cell is pressed and shows a semitransparent mine, which is basically a mine in \ColorTagShade{};
        \item[n] If \meta{info} is \meta{n}, then the cell is pressed and shows a blasted mine, with \ColorTagFail{} for background;
        \item[e] If \meta{info} is \meta{e}, then the cell is pressed and shows a mistakenly flagged mine, which is a cross in \ColorTagFail{} on top of a mine;
        \item[-] If \meta{info} is \meta{-}, then the cell is covered and has no content. A covered cell is drawn with the light side and dark side first. The widths of these sides are both 2pt, with \ColorTagLight{} and \ColorTagShade{} respectively to create a 3D effect. Then, the background is filled with \ColorTagZero{}. Note that the center of this cell is (8.5, -8.5);
        \item[(?)] If \meta{info} is any other \textbf{single} character, then the cell is covered and shows the character centered at (8, -8);
        \item[(*)] If \meta{info} contains \textbf{multiple} characters, the string will be shown in \tn[no-index]{tiny} font.
    \end{optdesc}
    The details of cell elements are in Fig. \ref{fig-flag-mine} and Fig. \ref{fig-cellup-celldown}.
}
\def\TransFigureFlagMineNumCaption{Sketch of the flag, the mine and the numbers}
\def\TransFigureCellUpDownCaption{Covered cell and revealed cell (the centers are marked)}

\def\TransSubsectionFrameElements{Frame elements}
\def\TransCommandBorderContent{Draw the cell-sized frame. \meta{t}, \meta{l}, \meta{b}, \meta{r} represent the top, left, bottom, right frame respectively. If two adjacent sides are drawn at the same time, the corner in between will also be drawn. The frame consists of 3 layers. The middle layer is in \ColorTagZero{} and 6pt wide. The inner and outer layer are 3pt wide, using \ColorTagLight{} and \ColorTagBorder{} to create a 3D effect similarly as a covered cell. The outer edge of the left and top highlights are enclosed with \ColorTagBorder{}. The default argument is \meta{-}, which means drawing the full frame.}
\def\TransFigureBorderCaption{The full frame in effect}

\def\TransSubsectionModules{Modules}
\def\TransSubsectionModulesContent{This section contains the modules provided by \pkg{tikz-minesweeper}. In general, it's recommended to only use these commands unless there is specific customization needed.}
\def\TransCommandRowContent{Draw multiple cells from left to right in the \meta{r}-th row. \meta{seq:info} is a sequence containing \meta{info}, where each element in the sequence serves as the third parameter of \tn{cell}. To display multiple characters in a cell, these characters should be surrounded by curly braces. For example, the length of the sequence \meta{1\{23\}4} is 3, and the elements in this sequence are \meta{1}, \meta{23}, and \meta{4} respectively.}
\def\TransCommandColContent{Draw multiple cells from top to bottom in the \meta{c}-th column, and the syntax is the same as \tn{row}.}
\def\TransCommandBoardContent{Draw the border with \meta{r} rows and \meta{c} columns. \meta{t}, \meta{r}, \meta{b}, \meta{l} share the same syntax as \tn{border}. By default, the board will be surrounded by all borders. If the \meta{-} argument is used, the borders in specific directions will be removed, and this argument can be customized. The parser only parses the position of the last \meta{-} option. For example, \tn[no-index]{border[-br-tl-lb-b]} will be parsed as \tn[no-index]{border[-b]}. The parser does not parse repeated parameters. For example, \tn[no-index]{border[ttblbtlb]} will be parsed as \tn[no-index]{border[tlb]}. The argument \meta{x} indicates whether to display the coordinates, which are hidden by default.}
\def\TransCommandBoardExtraNoteContent{During the process of drawing borders, this command will automatically use the clipping utility provided by \pkg{tikz}, based on the border type. Moreover, this command will draw the outer border with \ColorTagBorder{} and a thickness of 0.2pt. Additionally, this command will pre-draw the neighboring regions of the same color to avoid rendering issues caused by adjacent areas of the same color.}
\def\TransFigureBoardOptionCaption{Examples for different border arguments}
\def\TransFigureBoardCoordinateCaption{Examples for displaying coordinates}
\def\TransCommandColorCellContent{Color the cells in a sequence of areas \meta{seq: pos} with the \meta{color}. The \meta{color} argument can only accept color types that are accepted by \tn{cell}. However, unlike \tn{cell}, which only supports a single color, the \meta{color} argument supports up to four different colors. The colors will be filled counterclockwise starting from the top of a cell. \meta{seq: pos} defines a sequence of positions with the following format:
    \begin{itemize}
        \item Each element in the sequence is separated by a semicolon (;);
        \item For each element, the row range and column range are separated by a comma (,);
        \item For each range, if it contains \textbf{only one} character, the starting and ending positions are the same; otherwise, these positions need to be separated by a hyphen (-).
    \end{itemize}
}
\def\TransFigureColorCellCaption{Examples for colored cells}

\def\TransSectionImplementation{Implementation}
\def\TransMacroDependencyContent{Define the dependencies of \pkg{tikz-minesweeper}, which are \LaTeXiii{} and \pkg{tikz}.}
\def\TransMacroColorSchemeContent{Define the color scheme of \pkg{tikz-minesweeper}. Customized color schemes can be defined similarly in practice.}
\def\TransMacroUnitContent{Define the basic unit of \pkg{tikz-minesweeper}, which is 1pt.}
\def\TransMacroEnableLaTeXiiiEnvContent{Enable the \LaTeXiii{} environment.}
\def\TransMacroVariableDefinitionContent{Define the variables in \pkg{tikz-minesweeper}.}
\def\TransMacroFlagContent{Draw the flag.}
\def\TransMacroMineContent{Draw the mine.}
\def\TransMacroMineShadedContent{Draw the shaded mine. Note that the color of a shaded mine is coupled with other colors in the package. Please draw shaded mines with caution if different color schemes are applied.}
\def\TransMacroCellUpContent{Draw the covered cell.}
\def\TransMacroCellDownContent{Draw the revealed cell.}
\def\TransMacroCellFailContent{Draw the failed cell by clicking a mine.}
\def\TransMacroMisflagFailContent{Draw the failed cell by misflaging.}
\def\TransMacroCellNumContent{Use \ColorTagOne{} to \ColorTagEight{} to draw the given vectorized numbers from 1 to 8. It accepts one parameter \meta{\#1}. The parameter \meta{\#1} only accepts characters from \meta{12345678}. Any other value passed will be ignored.}
\def\TransMacroCellColoredContent{Draw the colored revealed cell. It accepts one parameter \meta{\#1}, which only accepts characters from \meta{rgbcyov}. Any other value passed will be ignored. The maximum length of parameter \meta{\#1} is 4. If the length of the parameter is less than 4, it will be padded on the left/right side with the same character. If the length of the parameter is greater than 4, it will be truncated to 4 characters.}
\def\TransMacroCellContent{Draw the cell, accepting three parameters \meta{\#1}, \meta{\#2}, \meta{\#3}. The cell will be drawn at coordinates (\meta{\#3}, \meta{\#2}), and the content of the cell is determined by \meta{\#1}. \meta{\#1} can accept any parameter. The function sequentially checks if \meta{\#1} is a numeric parameter \meta{012345678}, a color parameter \meta{rgbcyov}, a mine or flag parameter \meta{fmsne-}, a single character parameter excluding the mentioned characters, or a multi-character parameter.}
\def\TransMacroLineContent{Draw the row or column of cells, accepting three parameters \meta{\#1}, \meta{\#2}, \meta{\#3}. \meta{\#1} represents the row number or column number, \meta{\#2} represents the content of the grids, and \meta{\#3} determines whether to draw the grids sequentially in a row (\meta{r}) or in a column (\meta{c}). Other parameters will not be drawn.}
\def\TransMacroRowColMarkerContent{Draw row and column coordinates.}
\def\TransMacroGapFillerContent{Due to rendering issues, when drawing a board, there may be gaps between adjacent areas of the same color. This function can pre-fill these gaps.}
\def\TransMacroBorderContent{Draw the border, accepting three parameters \meta{\#1}, \meta{\#2}, \meta{\#3}. \meta{\#1} represents the border type parameter, \meta{\#2} represents the number of rows, and \meta{\#3} represents the number of columns. The parameter \meta{\#1} accepts values from \meta{-tlbr} only.}
\def\TransMacroBoardContent{Draw the board, accepting three parameters \meta{\#1}, \meta{\#2}, \meta{\#3}. \meta{\#1} represents the border type parameter, \meta{\#2} represents the number of rows, and \meta{\#3} represents the number of columns. The parameter \meta{\#1} accepts values from \meta{-tlbrx} only.}
\def\TransMacroColorCellContent{Color the cells in multiple areas, which accepts two parameters \meta{\#1} and \meta{\#2}. The parameter \meta{\#1} can only accept colors that are accepted by \cs[no-index]{msweeper_cellcolored:n}. The parameter \meta{\#2} is used to pass multiple regions to be drawn.}
\def\TransMacroUserAPIContent{Provide user APIs without calling \LaTeXiii{} directly.}
\def\TransMacroDisableLaTeXiiiEnvContent{Disable the \LaTeXiii{} environment.}
